%-------------------------------------------------------------------------------
% 
\chapter{Introduction}
%-------------------------------------------------------------------------------
An intelligent being or machine solving any kind of a problem needs knowledge to which it can apply its intelligence while coming up with an appropriate solution. This is especially true for the knowledge-driven AI systems which constitute a significant fraction of general AI research. For these applications, getting and formalizing the right amount of knowledge is crucial. This knowledge is acquired by some sort of Knowledge Acquisition (KA) process, which can be manual, automatic or semi-automatic. Knowledge acquisition, using an appropriate representation and subsequent knowledge maintenance are two of the fundamental and as-yet unsolved challenges of AI. Knowledge is still expensive to retrieve and to maintain. This is becoming increasingly obvious, with the rise of chat-bots and other conversational agents and AI assistants. The most developed of these (Siri, Cortana, Google Now, Alexa), are backed by huge financial support from their producing companies, and the lesser-known ones still result from 7 or more person-years of effort by individuals
\todo:{Finish}

\section{Contributions}
This section gives an overview of scientific and other contributions of this thesis to the knowledge acquisition approaches.

\subsection{Novel Approach Towards Knowledge Acquisition}
novel knowledge acquisition approach that uses logical inference over context and prior knowledge to automatically construct precisely targeted natural language crowdsourcing tasks (“questions”) for the right audience at the right moment. The process that drives knowledge acquisition is, at the same time, more directly addressing the user’s information needs based on the user context and the responses of other users who have been in similar contexts. The newly acquired knowledge is immediately utilized to construct better or more detailed questions and thus drive the KA process further

\subsection{A Shift From NL Patterns to Logic in Conversational Agents}
2.	A shift from NL-pattern-driven conversational agents with a knowledge base to remember facts, to knowledge driven agents where NL patterns only support the conversion from language into logical form, but do not directly drive the conversation.

\subsection{Knowledge Acquisition Platform Implementation as Technical Contribution }
3.	A technical (system) implementation of the above listed two contributions as a working real-world prototype which shows the feasibility of the approach and a way to connect many independent and complex sub-systems. Sensor data, natural language, inference engine, huge pre-existing knowledge base, textual patterns and crowdsourcing mechanisms are connected and interlinked into a coherent interactive application.