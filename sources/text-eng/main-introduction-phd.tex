%--------------------------------------------------------------------------------------------------
% 
\chapter{Introduction}
%--------------------------------------------------------------------------------------------------
An intelligent being or machine solving any kind of a problem needs knowledge to which it can apply its intelligence in coming up with an appropriate solution. This is especially true for the knowledge-driven AI systems which constitute a significant fraction of general AI research. For these applications, getting and formalizing the right amount of knowledge is crucial. This knowledge is acquired by some sort of Knowledge Acquisition (KA) process, which can be manual, automatic or semi-automatic. Knowledge acquisition, using an appropriate representation and subsequent knowledge maintenance are two of the fundamental and as-yet unsolved challenges of AI. Knowledge is still expensive to retrieve and to maintain. This is becoming increasingly obvious, with the rise of chat-bots and other conversational agents and AI assistants. The most developed of these (Siri, Cortana, Google Now, Alexa), are backed by huge financial support from their producing companies, and the lesser-known ones still result from 7 or more person-years of effort by individuals