%-------------------------------------------------------------------------------
% 
\chapter{Introduction}
%-------------------------------------------------------------------------------
An intelligent being or machine solving any kind of a problem needs knowledge to which it can apply its intelligence while coming up with an appropriate solution. This is especially true for the knowledge-driven AI systems which constitute a significant fraction of general AI research. For these applications, getting and formalizing the right amount of knowledge is crucial. This knowledge is acquired by some sort of Knowledge Acquisition (KA) process, which can be manual, automatic or semi-automatic. Knowledge acquisition using an appropriate representation and subsequent knowledge maintenance are two of the fundamental and as-yet unsolved challenges of AI. Knowledge is still expensive to retrieve and to maintain. This is becoming increasingly obvious, with the rise of chat-bots and other conversational agents and AI assistants. The most developed of these (Siri, Cortana, Google Now, Alexa), are backed by huge financial support from their producing companies, and the lesser-known ones still result from 7 or more person-years of effort by individuals
\todo:{\hl{Finish}}

Knowledge acquisition and subsequent knowledge maintenance, are two of the fundamental and as-yet not-completely-solved challenges of Artificial Intelligence (AI).

\hl{We propose and implement novel approach to automated knowledge acquisition using the user context obtained from a mobile device and knowledge based conversational crowdsourcing. The resulting system named Curious Cat has a multi objective goal, where KA is the primary goal, while having an intelligent assistant and a conversational agent as secondary goals. The aim is to perform KA effortlessly and accurately while having a conversation about concepts which have some connection to the user, allowing the system (or the user) to follow the links in the conversation to other connected topics. We also allow to lead the conversation off topic and to other domains for a while and possibly gather additional, unexpected knowledge. For illustration see the example conversation sketch in Table I, where topic changes from a specific restaurant to a type of dish. In this example case, the conversation is started by the system when user stays at the same location for 5 minutes.}

\section{Scientific Contributions}
This section gives an overview of scientific and other contributions of this thesis to the knowledge acquisition approaches.

\subsection{Novel Approach Towards Knowledge Acquisition}
Traditionally KA (knowledge acquisition) approach focuses on one type of acquisition process, which can be either Labor, Interaction, Mining or Reasoning\parencite{Zang2013}. In this thesis we propose a novel, previously untried approach that intervenes all aforementioned types with current user context and crowdsourcing into a coherent, collaborative and autonomous KA system. It uses existing knowledge and user context, to automatically deduce and detect  missing or unconfirmed knowledge(reasoning) and uses this info to generate crowdsourcing tasks for the right audience at the right time(labor). These tasks are presented to users in natural language (NL) as part of the contextual conversation (interaction) and the answers parsed (mining) and placed into the KB after consistency checks(reasoning). The approach contribution can be summed up as a) definition of the framework for autonomous and collaborative knowledge acquisition with the help of contextual knowledge (\hl{chapter X}), and b) demonstrate and evaluate the contributions of contextual knowledge and approach in general \hl{chapter X}.

\subsection{Knowledge Acquisition Platform Implementation as Technical Contribution }
Implementation of the KA framework as a working real-world prototype which shows the feasibility of the approach and a way to connect many independent and complex sub-systems. Sensor data, natural language, inference engine, huge pre-existing knowledge base (Cyc)\parencite{Lenat1995}, textual patterns and crowdsourcing mechanisms are connected and interlinked into a coherent interactive application (\hl{Chapter X}).

\subsection{A Shift From NL Patterns to Logical Knowledge Representation in Conversational Agents}
Besides the main contributions presented above, one aspect of the approach introduces a shift in the way how conversational agents are being developed. Normally the approach is to use textual patterns and corresponding textual responses, sometimes based on some variables, and thus encode the rules for conversation. As a consequence of natural language interaction, the proposed KA framework is in some sense a conversational agent which is driven by the knowledge and inference rules and uses patterns only for conversion from NL to logic. This shows promise as an alternative approach to building non scripted conversational engines (\hl{Chapter X}).

\section{Thesis structure}
The rest of the thesis is structured in to chapters covering specific topics. \hl{Chapter X} introduces\todo{...Finish}