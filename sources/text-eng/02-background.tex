%-------------------------------------------------------------------------------
\chapter{Background and Related Work}

In this chapter we will give an overview of approaches and related works on
broader knowledge acquisition research field, information extraction, 
crowdsourcing and geo-spatial context mining. 

Knowledge Acquisition has been addressed from different perspectives by many 
researchers in Artificial Intelligence over decades, starting already in 1970 
as a sub-discipline of AI research, and since then resulting in a big number of 
types and implementations of approaches and technologies/algorithms. The 
difficulty of acquiring and maintaining the knowledge was soon noticed and was 
coined as \emph{Knowledge Acquisition Bottleneck} in 
1977\parencite{Feigenbaum1977}. In more recent survey of KA approaches 
\parencite{Zang2013}, authors categorize all of the KA approaches into four main
groups, regarding the source of the data and the way knowledge is acquired:
\begin{itemize}
	\item \emph{Labour Acquisition.} This approach uses human minds as the 
    knowledge source. This usually involves human (expert) ontologists manually 
    entering and encoding the knowledge.
	\item \emph{Interaction Acquisition.} As in Labour Acquisition, the source 
    of the knowledge is coming from humans, but in this case the KA is wrapped 
    in a facilitated interaction with the system, and is sometimes implicit 
    rather than explicit.
	\item \emph{Reasoning Acquisition.} In this approach, new knowledge is 
    automatically inferred from the existing knowledge using logical rules and 
    machine inference.
	\item \emph{Mining Acquisition.} In this approach, the knowledge is 
    extracted from some large textual corpus or corpora.
\end{itemize}

We believe this categorization most accurately reflects the current state of 
machine (computer) based knowledge acquisition, and we decided to use the same 
classification when structuring our related work, focusing more on closely 
related approaches and extending where necessary. According to this 
classification, our work presented in this thesis, fits into a hybrid approach 
combining all four groups, with main focus on interaction and reasoning. We 
address the problem by combining the labour and interaction acquisition (users 
answering questions as part of NL interaction aimed at some higher level goal, 
such as helping the user with various tasks), adding unique features of using 
user context and existing knowledge in combination with reasoning to produce a 
practically unlimited number of potential interaction acquisition tasks, going 
into the field of crowd-sourcing by sending these generated tasks to many users 
simultaneously.

\todo{Fix this, reference to chapters instead to specifici works.} 
Previous works that can compare with our solution is divided into the systems 
that exploit existing knowledge (generated anew during acquisition or 
pre-existing from before in other sources) \parencite{Singh2002a,Witbrock2003,
Forbus2007,Kvo2010,Sharma2010,Mitchel2015}, reasoning \parencite{Witbrock2003,
Speer2007,Speer2008,Kuo2010}, crowdsourcing \parencite{Singh2002,Speer2009, 
Kuo2010, Pedro2012a, Pedro2013}, acquisition through interaction 
\parencite{Speer2009,Pedro2012,Pedro2013}, acquisition through labour(\hl{add, 
probably rather refer to subsections}) \parencite{} and natural language 
conversation\parencite{Pedro2012, Speer2007,Speer2009, Witbrock2003,Kuo2010}.

\hl{Test referencing table} (see \tablename~\ref{tab:related}).

\begin{landscape}
	\begin{table}[htb]
	\caption{Structured overview of related KA systems}
	\label{tab:related}
	\centering
	\begin{tabular}{lclcccccc}
		\hline
		System & Parent & Reference & Category & Source & Representation & Prior K. &  Crowds. & Context \\
		\hline
		Cyc project (Cycorp) & / & \parencite{Lenat1995} & Labour & K. Exp. & CycL & / & / & / \\
		ThoughtTrasure(Signiform) & / & \parencite{Mueller2003} & Labour & K. Exp. & LAGS & / & / & / \\
		HowNet (Keen.) & / & \parencite{Dong2010} & Labour & K. Exp. & KDML & / & / & / \\
		OMCS (MIT) & / & \parencite{Singh2002a} & Labour & Public & ConceptNet & / & \checkmark & / \\
		KRAKEN (Cycorp) & Cyc & \parencite{Panton2002a} & Interaction & D. Exp & CycL & \checkmark & / & / \\
		UIA (Cycorp) & Cyc & \parencite{Witbrock2003UIA} & Interaction & D. Exp & CycL & \checkmark & / & / \\
		Factivore (Cycorp) & Cyc & \parencite{Witbrock2005} & Interaction & D. Exp & CycL & \checkmark & / & / \\
		Predicate Populator (Cycorp) & Cyc & \parencite{Witbrock2005} & Interaction & D. Exp & CycL & \checkmark & / & / \\
		CURE (Cycorp) & Cyc & \parencite{Witbrock2010} & Interaction & D. Exp & CycL & \checkmark & / & / \\
		OMCommons (MIT) & OMCS & \parencite{Speer2007} & Interaction & Public & ConceptNet & \checkmark & \checkmark & / \\
		Freebase (Metaweb/Google) & / & \parencite{Bollacker2008} & Interaction & Public & RDF & / & / & / \\
		20 Questions (MIT) & OMCS & \parencite{Speer2009} & Game & Public & ConceptNet & / & / & / \\
		Verbosity (CMU) & OMCS & \parencite{Speer2009} & Game & Public & ConceptNet & / & / & / \\
		\hline
	\end{tabular}
\end{table}
\end{landscape}

\section{Labour Acquisition}
\label{section:LabourAcquisition}
This category consists of KA approaches which rely on explicit human work to 
collect the knowledge. A number of expert (or also untrained) ontologists or 
knowledge engineers is employed to codify the knowledge by hand into the given 
knowledge representation (formal language). Labour acquisition is the most 
expensive acquisition type, but it gives a high quality knowledge. It is often a
crucial initial step in other KA types as well, since it can help to have some 
pre-existing knowledge to be able to check the consistency of the newly acquired
knowledge. Labour Acquisition is often present in other KA types, even if not 
explicitly mentioned, since it is implicitly done when defining internal 
workings and structures of other KA processes. While we checked other well 
known systems that are result of Labour Acquisition, Cyc (mentioned below) is 
the most comprehensive of them and was picked as a starting point and main 
background knowledge and implementation base for this work.

\emph{Cyc.} The most famous and also most comprehensive and expensive knowledge 
acquired this way, is Cyc KB, which is part of Cyc AI system 
\parencite{Lenat1995}. It started in 1984 as a research project, with a premise 
that in order to be able to think like humans do, the computer needs to have 
knowledge about the world and the language like humans do, and there is no other
way than to teach them, one concept at a time, by hand. Since 1994, the project 
continued through Cycorp Inc. company, which is still continuing the effort. 
Through the years Cyc Inc. employed computer scientists, knowledge engineers, 
philosophers, ontologists, linguists and domain experts, to codify the knowledge
in the formal higher order logic language CycL \parencite{Matuszek2006a}. As of 2006
\parencite{Matuszek2006}, the effort of making Cyc was 900 non-crowdsourced 
human years which resulted in 7 million assertions connecting 500,000 terms and 
17,000 predicates/relations \parencite{Zang2013}, structured into consistent 
sub-theories (Microtheories) and connected to the Cyc Inference engine and 
Natural Language generation. Since the implemtentation of our approach is based
on Cyc, we give a more detailed description of the KB and its connected systems
in \autoref{section:Cyc} on page \pageref{section:Cyc}. Cyc Project is still 
work in progress and continues to live and expand through various research and
commercial projects.

\emph{ThoughtTreasure.} Approximately at the same time(1994) as Cyc Inc. company
was formed, Eric Mueller started to work on a similar system, which was inspired
by Cyc and is similar in having a combination of common sense knowledge concepts
connected to their natural language presentations. The main differentiator from 
Cyc is, that it tries to use simpler representation compared to first-order 
logic as is used in Cyc. Additionally, some parts of ThoughtTreasure knowledge 
can be presented also with finite automata, grids and scripts 
\parencite{Mueller1999,Mueller2003}. In 2003 the knowledge of this system 
consisted of 25,000 concepts and 50,000 assertions. ThoughtTreasure was not so 
successfull as Cyc and ceased all developments in 2000 and was open-sourced on 
Github in 2015. \hl{link as footnote}.

\emph{HowNet} started in 1999 and is an on-line common-sense knowledge base 
unveiling inter-conceptual relationships and inter-attribute relationships of 
concepts as connoting in lexicons of the Chinese and their English equivalents. 
As of 2010 it had 115,278 concepts annotated with Chinese representation, 
121,262 concepts with English representation, and 662,877 knowledge base records
including other concepts and attributes \parencite{Dong2010}. HowNet knowledge 
is stored in the form of concept relationships and attribute relationships and 
is formally structured in KDML (Knowledge Database Mark-up Language), consisting
of concepts (called semens in KDML) and their semantic roles.
 
\emph{Open Mind Common Sense (OMCS)} is a crowdsourcing knowledge acquisition 
project that started in 1999 at the MIT Media Lab\parencite{Singh2002a}. 
Together with initial seed and example knowledge, the system was put online with
a knowledge entry interface, so the entry was crowd-sourced and anyone 
interested could enter and codify the knowledge. OMCS supported collecting 
knowledge in multiple languages. It's main difference from the systems described
above (Cyc, HowNet, ThoughtTreasure) is, that it used deliberate crowdsourcing
and that it's knowledge base and representation is not strictly formal logic, 
but rather inter-connected pieces of natural language statements. As of 2013 
\parencite{Zang2013}, OMCS produced second biggest KB after Cyc, consisting of 
English (1,040,067 statements), Chinese (356,277), Portuguese (233,514), 
Korean (14,955), Japanese (14,546), Dutch (5,066), etc. Initial collection was
done by specifying 25 human activities, where each activity got it's own user 
interface for free form natural language entry and also pre-defined patterns 
like "A hammer is for \underline{\hspace{1.5cm}}", where participants can enter
the knowledge. Although OMCS started to build KB from scratch it shares a 
similarity to our CC system in a sense that it is using crowd-sourcing and also
natural language patterns with empty slots to fill in missing parts. OMCS was
later used in many other KA approaches as a prior knowledge, similar way as we 
use Cyc. After a few versions, OMCS was taken from public access and merged with
multiple KBs and KA approaches into an ConceptNet 
KB\footnote{http://conceptnet.io/} \parencite{Speer2016}, which is now (in 2017)
part of Linked Open Data (LOD) and maintained as open-source project.
 
\section{Interaction Acquisition}
Similarly as with Labour KA, interaction Acquisition gets the knowledge from 
human minds, but in this case the acquisition is an intended side effect, while
users are interacting with the software as part of some other activity/task, or
as part of a motivation scheme, such as knowledge acquisition games. Besides 
games, the interaction could be some other user interface for solving specific
tasks, or a Natural Language Conversation. This type of acquisition is most 
strongly correlated with the approach described in this thesis, since Curious 
Cat uses points (gaming), to motivate users and it interacts with user in NL, 
while discussing various topics (concepts). It uses the conversation to set up
the context and acquire (remember) user's responses and places them properly in
to the KB. Sometimes the acquired knowledge is paraphrased and presented back to
user to show the 'understanding', which was first tried in OSMC (
\autoref{section:LabourAcquisition}, \parencite{Singh2002b}), but there only in
non-conversational way as part of the input forms.
 
\subsection{Interactive User Interfaces}
Interactive user interfaces are the most common representation of interaction 
acquisition, where the user interface is constructed in a way to help user enter
the data and thus make the acquisition much faster and cheaper. Historically, 
these systems were developed to help the labour acquisition systems, or on top
of them, after parent systems reached some sort of maturity and initial 
knowledge stability. This is the reason why all of these systems rely or are 
build on top of labour acquisition (\autoref{section:LabourAcquisition}) or 
mining acquisition (\autoref{section:MiningAcquisition}) systems.

\emph{KRAKEN} system was a knowledge entry tool which allows domain experts to
make meaningful additions to CYC knowledge base, without the training in the 
areas of artificial intelligence, ontology development, or knowledge
representation\parencite{Panton2002a}. It was developed as part of DARPA's
Rapid Knowledge Formation (RKF) project in 2000. As its goal was to allow
knowledge entry to non-trained experts, it started to use natural language 
entry and is as this, a first pre-cursor to Curious Cat system and a seed idea
for it. It consists of creators, selectors, modifiers of Cyc KB building blocks,
tools for consistency checks and tools for using existing knowledge to infer new
things to ask. This tool, together with it's derived solutions was later 
re-written and integrated into Cyc as CURE system (see below). While KRAKEN and
later CURE already used Natural Language generation and parsing, and started 
with the idea of natural language dialogue for doing the KA, the interaction, it
was missing user context (user's had to select or search the concept of 
interest), and also crowdsourcing aspects. Kraken was also missing rules for
explicit question asking. The questions were all related to the selected concept
and given as a list of natural language forms.

\emph{User Interaction Agenda (UIA)} was a web  based user interface for KRAKEN
KA tool\parencite{Panton2002a,Witbrock2003UIA}. It worked inside a browser and 
it worked as responsive web-app (in 2001) by automatically triggering refresh 
functionality of the browser. It consisted of a menu of tools that is organized
according to the recommended steps of the KE process, text entry box (query, 
answer, statement), center screen for the main interaction with the current 
tool, and a summary with a set of colored steps needed to complete current 
interaction. Similarly as KRAKEN itself, this interface was later improved
and integrated into main Cyc system as part of CURE tool. 

\emph{Factivore} was a Java Applet user interface for an extended KRAKEN system,
meant for quick facts entering \parencite{Witbrock2005}. On the back-end it used
the same mechanisms and logical templates, while in the front-end it only
allowed facts entering, as opposed to UIA, which also allowed rules (which
ended up as not being useful).

\emph{Predicate Populator} is a similar tool as \emph{Factivore}, which instead
of only collecting instances, allows to add general knowledge about classes. For
example, instead of describing facts for a specific restaurant, it can collect
general knowledge that is true for all restaurants \parencite{Witbrock2005}. The
context of the KA in this case, is given by class concept, a predicate and a 
web-site which is parsed into CycL concepts. These are then filtered out if they
do not match argument constraints of the predicate and then shown to user for 
selection. As part of the validation, this tool had some problems with correctly
acquired knowledge. One of the proposed solutions (never implemented), was to
start using volunteers to vote about the correctness. This is already a 
pre-cursor idea for crowd-sourced voting mechanisms that we used in Curious Cat.

\emph{Freebase} started in 2007\parencite{Bollacker2008} and was a large (mostly
instance based) crowd-sourced graph database for structured general human 
knowledge. Initially it was acquired from multiple public sources, mostly 
Wikipedia. The initial seed was then constantly updated and corrected by the 
community. On the user interface side, Freebase provides an AJAX/Web based 
UI for humans and an HTTP/JSON based API for software access. For finding
knowledge and also software based editing, it uses Metaweb Query Language 
(MQL). A company behind freebase was bought by Google in 2010 and incorporated
into a Google Knowledge Graph. In 2016 Freebase was incorporated into the 
Wikidata platform and shut down by Google and is no longer maintained.

\emph{OMCommons (Open Mind Commons)} is an interactive interface to OMCS which
can respond with a feedback to user answers and maintain dialogue 
\parencite{Speer2007}. This is similar approach as we do with Curious Cat and
shows understanding of the knowledge users enter. The mechanisms behind is
by using inference engine to make analogical inferences based on the existing 
knowledge and new entry. Then it generates some relevant questions and asks 
user to confirm them. For example, as given from the original paper, 
\emph{OMCommons} asks: "A bicycle would be found on the street. Is this common 
sense?". This is then displayed to the user with the justification for the 
question: "A bicycle is similar to a car. I have been told that a car would 
be found on the street". Users then click on "Yes/No" buttons to confirm or
reject the inferred statement. The interactive interface also allows its users 
to refine the knowledge entered by other users and see the ratings. Users can 
also explore what new inferences are result of their new contributions.

\subsection{Games}
Games are a specific sub-section of interaction acquisition, where the actual
acquisition is hidden or transformed into much more enjoyable process, 
maximizing the entertainment of the users.

\emph{20Q (20 Questions)} is a game with intentional knowledge acquisition task
which focuses to the most salient properties of concepts. The game itself is
a standard 20 questions game which aims to make one player figure out the 
concept of discussion by asking yes/no questions and then infer from the 
answers what the concept could be. The only difference is that the player which
is asking is a computer based on OMCS knowledge base. It generates questions in
NL, and according to what a player answers, it attempts to guess the concept.
To decide what questions to ask, it uses statistical classification methods
\parencite{Speer2009}, to discover the most informative attributes of concepts
in OMCS KB. After the user answers all the questions, including whether the
detected concept was right or not, the concept and the answers will be assigned
to proper cluster and thus the characteristics of the object are learned.

\subsection{Interactive Natural Language Conversation}
dada. Describe also which Games goes under this category. But this should mostly be about chatbots

\section{Reasoning Acquisition}
adad dada

\section{Mining Acquisition}
\label{section:MiningAcquisition}
adad

\section{Acquisition with the help of existing knowledge}
adad

\section{Crowdourcing Acquisition}
adad

\section{Acquistion of Geospatial Context}
adad
