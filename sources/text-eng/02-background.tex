%-------------------------------------------------------------------------------
\chapter{Background and Related Work}

In this chapter we will give an overview of approaches and related works on broader knowledge acquisition research field, information extraction, crowdsourcing and geo-spatial context mining. 

Knowledge Acquisition has been addressed from different perspectives by many researchers in Artificial Intelligence over decades, starting already in 1970 as a sub-discipline of AI research (\hl{Feigenbaum-economicPhd}), and since then resulting in a big number of types and implementations of approaches and technologies/algorithms. In more recent survey of KA approaches \parencite{Zang2013}, authors categorize all of the KA approaches into four main groups, regarding the source of the data and the way knowledge is acquired:
\begin{itemize}
	\item \emph{Labour Acquisition.} This approach uses human minds as the knowledge source. This usually involves human (expert) ontologists manually entering and encoding the knowledge.
	\item \emph{Interaction Acquisition.} As in Labour Acquisition, the source of the knowledge is coming from humans, but in this case the KA is wrapped in a facilitated interaction with the system, and is sometimes implicit rather than explicit.
	\item \emph{Reasoning Acquisition.} In this approach, new knowledge is automatically inferred from the existing knowledge using logical rules and machine inference.
	\item \emph{Mining Acquisition.} In this approach, the knowledge is extracted from some large textual corpus or corpora.
\end{itemize}

We believe this categorization most accurately reflects the current state of machine (computer) based knowledge acquisition, and we decided to use the same classification when structuring our related work, focusing more on closely related approaches and extending where necessary. According to this classification, our work presented in this thesis, fits into a hybrid approach combining all four groups, with main focus on interaction and reasoning. We address the problem by combining the labour and interaction acquisition (users answering questions as part of NL interaction aimed at some higher level goal, such as helping the user with various tasks), adding unique features of using user context and existing knowledge in combination with reasoning to produce a practically unlimited number of potential interaction acquisition tasks, going into the field of crowd-sourcing by sending these generated tasks to many users simultaneously.

\todo{Fix this, reference to chapters instead to specifici works.} 
Previous works that can compares to our solution is divided into the systems that exploit existing knowledge (generated anew during acquisition or pre-existing from before in other sources) \parencite{Singh2002a,Witbrock2003,Forbus2007,Kvo2010,Sharma2010,Mitchel2015}, reasoning \parencite{Witbrock2003,Speer2007,Speer2008,Kuo2010}, crowdsourcing \parencite{Singh2002,Speer2009, Kuo2010, Pedro2012a, Pedro2013}, acquisition through interaction \parencite{Speer2009,Pedro2012,Pedro2013}, acquisition through labour(\hl{add, probably rather refer to subsections}) \parencite{} and natural language conversation\parencite{Pedro2012, Speer2007,Speer2009, Witbrock2003,Kuo2010}.

\hl{Test referencing table} (see \tablename~\ref{tab:related}).

\begin{table}[htb]
	\caption{Structured overview of related KA systems}
	\label{tab:related}
	\centering
	\begin{tabular}{cccccccc}
		\hline
		System & Ref. & Cat. & Source & Repr. & PK &  CS & C \\
		\hline
		Cyc project (Cycorp) & \parencite{Lenat1995} & Labour & K. Exp. & CycL & / & / & / \\
		ThoughtTrasure(Signiform) & \parencite{Mueller2003} & Labour & K. Exp. & LAGS & / & / & / \\
		HowNet (Keen.) & \parencite{Dong2010} & Labour & K. Exp. & KDML & / & / & / \\
		OMCS (MIT) & \parencite{Singh2002} & Labour & G. Public & ? & / & \checkmark & / \\
		\hline
	\end{tabular}
\end{table}

\section{Labour Acquisition}
This category consists of KA approaches which rely on explicit human work to collect the knowledge. A number of expert (or also untrained) ontologists or knowledge engineers is employed to codify the knowledge by hand into the given knowledge representation (formal language). Labour acquisition is the most expensive acquisition type, but it gives a high quality knowledge. It is often a crucial initial step in other KA types as well, since it can help to have some pre-existing knowledge to be able to check the consistency of the newly acquired knowledge. Labour Acquisition is often present in other KA types, even if not explicitly mentioned, since it is implicitly done when defining internal workings and structures of other KA processes. While we checked other well known systems that are result of Labour Acquisition, Cyc (mentioned below) is the most comprehensive of them and was picked as a starting point and main background knowledge and implementation base for this work.

\emph{Cyc.} The most famous and also most comprehensive and expensive knowledge acquired this way, is Cyc KB, which is part of Cyc AI system \parencite{Lenat1995}. It started in 1984 as a research project, with a premise that in order to be able to think like humans do, the computer needs to have knowledge about the world and the language like humans do, and there is no other way than to teach them, one concept at a time, by hand. Since 1994, the project continued through Cycorp Inc. company, which is still continuing the effort. Through the years Cyc Inc. employed computer scientists, knowledge engineers, philosophers, ontologists, linguists and domain experts, to codify the knowledge in the formal higher order logic language CycL (\hl{Cyc Language}). As of 2006   \parencite{Matuszek2006}, the effort of making Cyc was 900 non-crowdsourced human years which resulted in 7 million assertions connecting 500,000 terms and 17,000 predicates/relations \parencite{Zang2013}, structured into consistent sub-theories (Microtheories) and connected to the Cyc Inference engine and Natural Language generation. Since the implemtentation of our approach is based on Cyc, we give a more detailed description of the KB and its connected systems in \autoref{section:Cyc} on page \pageref{section:Cyc}. Cyc Project is still work in progress and continues to live and expand through various research and commercial projects.

\emph{ThoughtTreasure.} Approximately at the same time(1994) as Cyc Inc. company was formed, Eric Mueller started to work on a similar system, which was inspired by Cyc and is similar in having a combination of common sense knowledge concepts connected to their natural language presentations. The main differentiator from Cyc is, that it tries to use simpler representation compared to first-order logic as is used in Cyc. Additionally, some parts of ThoughtTreasure knowledge can be presented also with finite automata, grids and scripts \parencite{Mueller1999,Mueller2003}. In 2003 the knowledge of this system consisted of 25,000 concepts and 50,000 assertions. ThoughtTreasure was not so successfull as Cyc and ceased all developments in 2000 and was open-sourced on Github in 2015.

\emph{HowNet} started in 1999 and is an on-line common-sense knowledge base unveiling inter-conceptual relationships and inter-attribute relationships of concepts as connoting in lexicons of the Chinese and their English equivalents . As of 2010 it had 115,278 concepts annotated with Chinese representation, 121,262 concepts with English representation, and 662,877 knowledge base records including other concepts and attributes \parencite{Dong2010}. HowNet knowledge is stored in the form of concept relationships and attribute relationships and is formally structured in KDML (Knowledge Database Mark-up Language), consisting of concepts (called semens in KDML) and their semantic roles.
 
\emph{Open Mind Common Sense (OMCS)} is a crowdsourcing knowledge acquisition project that started in 1999 at the MIT Media Lab. Together with initial seed and example knowledge, the system was put online with a knowledge entry interface, so the knowledge entry was crowd-sourced and anyone interested could enter and codify the knowledge. OMCS supported collecting knowledge in multiple languages. It's main difference from the systems described above (Cyc, HowNet, ThoughtTreasure) is, that it used deliberate crowdsourcing and that it's knowledge base and representation is not strictly formal logic, but rather inter-connected pieces of natural language statements. As of 2013 \parencite{Zang2013}, OMCS produced second biggest KB after Cyc, consisting of English (1,040,067 statements), Chinese (356,277), Portuguese (233,514), Korean (14,955), Japanese (14,546), Dutch (5,066), etc. Initial collection was done by specifying 25 human activities, where each activity got it's own user interface for free form natural language entry and also pre-defined patterns like "A hammer is for \underline{\hspace{1.5cm}}", where participants can enter the knowledge. Although OMCS started to build KB from scratch it shares a similarity to our CC system in a sense that it is using crowd-sourcing and also natural language patterns with empty slots to fill in missing parts. OMCS was later used in many other KA approaches as a prior knowledge, similar way as we use Cyc. After a few versions, OMCS was taken from public access and merged with multiple KBs and KA approaches into an ConceptNet KB\footnote{http://conceptnet.io/} \parencite{Speer2016}, which is now (in 2017) part of Linked Open Data (LOD) and maintained as open-source project.
 
\section{Interaction Acquisition}
Similarly as with Labour KA, interaction Acquisition gets the knowledge from human minds, but in this case the acquisition is an intended side effect, while users are interacting with the software as part of some other activity/task, or as part of a motivation scheme, such as knowledge acquisition games. Besides games, the interaction could be some other user interface for solving specific tasks, or a Natural Language Conversation. This type of acquisition is most strongly correlated with the approach described in this thesis, since Curious Cat uses points (gaming), to motivate users and it interacts with user in NL, while discussing various topics (concepts). It uses all the conversation to set up the context and acquire (remember) user's responses and places them properly in to the KB. Sometimes the acquired knowledge is paraphrased and presented back to user to show the 'understanding'. This had been tried to some extent in Z \parencite{Singh2002b}.
 

\subsection{Games}
adada
\subsection{Interactive User Interfaces}
adada

\subsection{Interactive Natural Language Conversation}
dada

\section{Reasoning Acquisition}
adad dada

\section{Mining Acquisition}
adad

\section{Acquisition with the help of existing knowledge}
adad

\section{Crowdourcing Acquisition}
adad

\section{Acquistion of Geospatial Context}
adad
