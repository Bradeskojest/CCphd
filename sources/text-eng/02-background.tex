%-------------------------------------------------------------------------------
\chapter{Background and Related Work}

In this chapter we will give an overview of approaches and related works on broader knowledge acquisition research field, information extraction, crowdsourcing and geo-spatial context mining. 

Knowledge Acquisition has been addressed from different perspectives by many researchers in Artificial Intelligence over decades, starting already in 1970 as a sub-discipline of AI research (\hl{Feigenbaum-economicPhd}), and since then resulting in a big number of types and implementations of approaches and technologies/algorithms. In more recent survey of KA approaches \parencite{Zang2013}, authors categorize all of the KA approaches into four main groups, regarding the source of the data and the way knowledge is acquired:
\begin{itemize}
	\item \emph{Labour Acquisition.} This approach uses human minds as the knowledge source. This usually involves human (expert) ontologists manually entering and encoding the knowledge.
	\item \emph{Interaction Acquisition.} As in Labour Acquisition, the source of the knowledge is coming from humans, but in this case the KA is wrapped in a facilitated interaction with the system, and is sometimes implicit rather than explicit.
	\item \emph{Reasoning Acquisition.} In this approach, new knowledge is automatically inferred from the existing knowledge using logical rules and machine inference.
	\item \emph{Mining Acquisition.} In this approach, the knowledge is extracted from some large textual corpus or corpora.
\end{itemize}

We believe this categorization most accurately reflects the current state of machine (computer) based knowledge acquisition, and we decided to use the same classification when structuring our related work, focusing more on closely related approaches and extending where necessary. According to this classification, our work presented in this thesis, fits into a hybrid approach combining all four groups, with main focus on interaction and reasoning. We address the problem by combining the labour and interaction acquisition (users answering questions as part of NL interaction aimed at some higher level goal, such as helping the user with various tasks), adding unique features of using user context and existing knowledge in combination with reasoning to produce a practically unlimited number of potential interaction acquisition tasks, going into the field of crowd-sourcing by sending these generated tasks to many users simultaneously.

\todo{Fix this, reference to chapters instead to specifici works.} 
Previous works that can compares to our solution is divided into the systems that exploit existing knowledge (generated anew during acquisition or pre-existing from before in other sources) \parencite{Singh2002a,Witbrock2003,Forbus2007,Kvo2010,Sharma2010,Mitchel2015}, reasoning \parencite{Witbrock2003,Speer2007,Speer2008,Kuo2010}, crowdsourcing \parencite{Singh2002,Speer2009, Kuo2010, Pedro2012a, Pedro2013}, acquisition through interaction \parencite{Speer2009,Pedro2012,Pedro2013}, acquisition through labour(\hl{add, probably rather refer to subsections}) \parencite{} and natural language conversation\parencite{Pedro2012, Speer2007,Speer2009, Witbrock2003,Kuo2010}.

\hl{Test referencing table} (see \tablename~\ref{tab:related}).

\begin{table}[htb]
	\caption{Structured overview of related KA systems}
	\label{tab:related}
	\centering
	\begin{tabular}{cccccccc}
		\hline
		System & Ref. & Cat. & Source & Repr. & PK &  CS & C \\
		\hline
		Cyc project (Cycorp) & \parencite{Lenat1995} & Labour & Knowledge Entry & CycL & / & / & / \\
		51 & 2234 & 97 & X & ? \\
		\hline
	\end{tabular}
\end{table}

\section{Labour Acquisition}
This category consists of KA approaches which rely on explicit human work to collect the knowledge. A number of expert (or also untrained) ontologists or knowledge engineers is employed to codify the knowledge by hand into the given knowledge representation (formal language). 

 \emph{Cyc.} The most famous and also most comprehensive and expensive knowledge acquired this way, is Cyc KB, which is part of Cyc AI system \parencite{Lenat1995}. It started in 1984 as a research project, with a premise that in order to be able to think like humans do, the computer needs to have knowledge about the world and the language like humans do, and there is no other way than to teach them, one concept at a time, by hand. Since 1994, the project continued through Cycorp Inc. company, which is still continuing the effort. Through the years Cyc Inc. employed computer scientists, knowledge engineers, philosophers, ontologists, linguists and domain experts, to codify the knowledge in the formal higher order logic language CycL (\hl{Cyc Language}). As of 2006   \parencite{Matuszek2006}, the effort of making Cyc was 900 non-crowdsourced human years which resulted in 7 million assertions connecting 500.000 terms and 17.000 predicates/relations \parencite{Zang2013}, structured into consistent sub-theories (Microtheories) and connected to the Cyc Inference engine and Natural Language generation. Since the implemtentation of our approach is based on Cyc, we give a more detailed description of the KB and its connected systems in \autoref{section:Cyc} on page \pageref{section:Cyc}.

\section{Interaction Acquisition}
adada
\subsection{Games}
adada
\subsection{Natural Language Conversation}
dada

\section{Reasoning Acquisition}
adad dada

\section{Mining Acquisition}
adad

\section{Acquisition with the help of existing knowledge}
adad

\section{Crowdourcing Acquisition}
adad

\section{Acquistion of Geospatial Context}
adad
