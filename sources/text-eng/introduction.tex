%-------------------------------------------------------------------------------
% 
\chapter{Introduction}
%-------------------------------------------------------------------------------
An intelligent being or machine solving any kind of a problem needs knowledge 
over which it can apply its intelligence when inferencing in order to be able
and come up with appropriate solutions. This is especially true for any 
knowledge-driven AI systems which constitute a significant fraction of general 
AI research. For these applications, getting and formalizing the knowledge that
can be trusted is crucial, but not yet straight forward to achieve on scale 
within accessible cost boundaries. Even in the case if it happens that future 
research will prove the knowledge based systems to be the dead end in the
evolution of AI, solving the Knowledge Acquisition 
Bottleneck\parencite{Wagner2006} will be one of the required and critical 
stepping stones that will need to be involved.

Such knowledge needs to be acquired by some sort of Knowledge Acquisition (KA) 
process, either manual, automatic or semi-automatic, converted into the
appropriate representation and subsequently maintained, to keep up with the
changes in the underlying domains. That this is costly is becoming 
increasingly obvious, with the rise of chat-bots and other conversational 
agents and AI assistants that became popular throughout the last few years. 
The most developed of these (Siri, Cortana, Google Now, Alexa, Bixby), are 
all backed by huge financial support from their producing companies, and the 
lesser-known ones are still result of 7 or more person-years of efforts by
individuals\parencite{Wilcox2011, Wallace2013}, or smaller companies (e.g.,
Josh AI\footnote{www.josh.ai}). Even while these systems may use statistical
and machine learning (e.g. deep learning) approaches, a substantial 
portion of their production effort lies in knowledge acquisition, which is 
sometimes hidden in (hand) coded rules for request/response patterns, execution
workflow constructions and corresponding actions.

Independently of the far goal of AI research (human like intelligence), which
is relected through latest instances of AI assistants, there are countless 
expert systems, databases and other software solutions that need maintenance
and additional semantics, and all suffer from the KA bottleneck. This is making
the solutions expensive, or unable to fully utilize the idea that they 
implement. The bottleneck is especially obvious for the systems that make use 
of the real-world data that is rapidly changing through time. Some of this can
be reflected through revenues of the crowdsourcing platforms (Amazon Mechanical
Turk, Crowdflower).

This dissertation deals with the KA bottleneck problem by developing a novel
approach to automated knowledge acquisition, using existing knowledge,
contextual crowdsourcing and natural language, to make the KA a side effect of
effortless interaction of users with the computer. The thesis aims to develop,
implement and evaluate the approach and do a step closer towards removing the
bottleneck part from the knowledge acquisition.

The approach implementation resulted in a real world mobile assistant 
application (called Curious Cat), which is able to use existing knowledge to 
automatically infer 
what knowledge is missing and then ask users for the answers, check for
the correctnes and consistency and then place the newly acquired answer on
top of the existing knowledge, to repeat the KA loop. It is
using the context obtained from mobile devices to ask the right users at the 
right time, about the right topic. 
The implenentation has a multi objective goal, 
where KA is the primary goal, and an intelligent assistant and conversational 
agent are secondary goals, making the KA effortless and accurate while having a 
conversation about concepts which have some connection to the user.

The evaluation of the approach shows that the knowledge collected this way is
mostly correct and consistent with the existing KB and thus shows the approach
is promising and could work also in other real-world scenarious.

\section{Scientific Contributions}
This section gives an overview of scientific and other contributions of this 
thesis to the knowledge acquisition approaches.

\subsection{Novel Approach Towards Knowledge Acquisition}
Traditionally KA (knowledge acquisition) approach focuses on one type of 
acquisition process, which can be either Labor, Interaction, Mining or 
Reasoning\parencite{Zang2013}. In this thesis we propose a novel, previously 
untried approach that intervenes all aforementioned types with current user 
context and crowdsourcing into a coherent, collaborative and autonomous 
KA system. It uses existing knowledge and user context, to automatically 
deduce and detect missing or unconfirmed knowledge(\emph{reasoning}) and uses 
this info to generate crowdsourcing tasks for the right audience at the right 
time(\emph{labor}). These "KA tasks" are presented to users in natural language
(NL) as part of the contextual conversation (\emph{interaction}) and the 
answers are parsed (\emph{mining}) and placed into the KB after consistency 
checks(\emph{reasoning}). The approach contribution can be summed up as

\begin{enumerate}
\item definition of the framework for autonomous and collaborative knowledge 
acquisition with the help of contextual knowledge (\autoref{chapter:approach}), 
\item demonstrate and evaluate the contributions of contextual knowledge and 
approach in general (chapters \ref{chapter:implementation} and 
\ref{chapter:evaluation}).
\end{enumerate}

\subsection{Knowledge Acquisition Platform Implementation as Technical 
Contribution }
Implementation of the KA framework as a working real-world prototype which 
shows the feasibility of the approach and a way to connect many independent and 
complex sub-systems. Sensor data, natural language, inference engine, 
huge pre-existing knowledge base (Cyc\footnote{Cyc is a registered trademark 
of Cycorp, Inc.})\parencite{Lenat1995}, textual patterns 
and crowdsourcing mechanisms are connected and interlinked into a coherent 
interactive application (\autoref{chapter:implementation}).

\subsection{A Shift From NL Patterns to Logical Knowledge Representation in 
Conversational Agents}
Besides the main contributions presented above, one aspect of the approach 
introduces a shift in the way how conversational agents are being developed. 
Normally the approach is to use textual patterns and corresponding textual 
responses, sometimes based on some variables, and thus encode the rules for 
conversation. As a consequence of natural language interaction, the proposed 
KA framework is in some sense a conversational agent which is driven by the 
knowledge and inference rules and uses patterns only for conversion from 
NL to logic. This shows promise as an alternative approach to building non 
scripted conversational engines (\autoref{chapter:approach}, 
\autoref{section:dialog}).

%section
\section{Thesis structure}
The thesis starts with a short introduction of the thesis, its main 
contributions and the knowledge acquisition topic it covers. 
Chapter \ref{chapter:background} presents the challenges that have to be overcome
or addressed by this work, and separate components required to construct the 
working approach.

Chapter \ref{chapter:related} presents an overview of the related work and
similar approaches from the area of knowledge acquisition.

Chapter \ref{chapter:approach} describes the general, implementation agnostic
knowledge acquisition approach, one of the main contributions of the dissertation.
The chapter starts with the overall architecture and then works it's way through
sections, explaining the approach and addressing problems highlited in
\autoref{chapter:background}. Each section defines required minimal logical
structures to be able to explain the approach. Definitions from the sections
build on top of each-other, ending in an explanatory knowlege base, and
a last section which shows through the concrete example, how the separate 
modules work together as a coherent KA platform.

Chapter \ref{chapter:implementation} presents our real world implementation of 
the proposed system. It describes which tools and algorithms we used for 
particular components of the whole approach.

Chapter \ref{chapter:evaluation} presents the results of the evaluation and
analysis of the implemented system and the knowledge it collected during the
duration of experiments.

Chapter \ref{chapter:conclusions} gives main remarks and conclusions about the
approach and its presented implementation. It also gives directions for future
research.
