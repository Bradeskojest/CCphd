%--------------------------------------------------------------------------------------------------
% 
\chapter{Equations and Measurement Units}
%--------------------------------------------------------------------------------------------------

\section{Equations}

Small equations are often written in-line (within the text), for example $j^\star = \sigma T^4$, while larger ones need to be displayed in the following way:
\begin{equation}
	\sigma = \frac{2 \pi^5 k^4}{15 c^2 h^3} = 5.6704 \times 10^{-8} J s^{-1} m^{-2} K^{-4}
	\label{eq:complex}
\end{equation}
All displayed equations need to be numbered so that they can be referenced later in the text (for example, Eq.~(\ref{eq:complex}) presents the Stefan's (or Stefan-Boltzmann) constant).

\section{Measurement Units}

The candidate can choose a standard for measurement units and has to consistently use it throughout the thesis. 