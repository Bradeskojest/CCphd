%-------------------------------------------------------------------------------
% 
\chapter*{Povzetek}
\pdfbookmark[0]{Povzetek}{Povzetek}
%-------------------------------------------------------------------------------

Zajemanje strukturiranega znanja je pomemben del raziskav s področja umetne
inteligence. Dostop do visoko kakovostnega znanja, strukturiranega v obliko, ki omogoča uporabo z računalniškimi programi, je pomemben dejavnik, ki pripomore k razvoju tudi drugih podpodročij znanosti, predvsem računalništva, strojnega učenja ter preostalih vej umetne inteligence. Gre za zahtevno nalogo, ki se jo
rešuje že nekaj desetletij s pomočjo mnogih metod, še vedno pa tudi ročnega vnašanja in kodiranja znanja v kompleksne baze podatkov s strani strokovnjakov
posamičnih področij. V zadnjih nekaj letih so se pojavili poizkusi zajemanja
znanja tudi s pomočjo množičenja, kjer gre za prestrukturiranje problema na
manjše in enostavnejše podprobleme, predstavljene tako, da jih namesto enega strokovnjaka lahko reši večja množica ne nujno strokovno usposobljenih oseb.

Kljub temu da je bilo odkritih, razvitih in testiranih mnogo sistemov za pridobivanje znanja s pomočjo množičenja in taki sistemi uspešno rešijo težavo pomanjkanja strokovnega kadra, še vedno ostajajo problemi visokih finančnih stroškov, težavne priprave nalog za množičenje, iskanja prave množice, konsistentnosti in kakovosti pridobljenega znanja.

To doktorsko delo rešuje omenjen primanjkljaj z vpeljavo in formalizacijo novega pristopa k zajemanju strukturiranega znanja, ki lahko znatno poveča kakovost pridobljenega znanja in posledično zmanjša stroške. To rešujemo z
avtomatskim generiranjem nalog za množičenje v naravnem jeziku, ki pridejo do množice uporabnikov kot vprašanja, na katera enostavno odgovarjajo. Pristop
uporablja obstoječo bazo strukturiranega znanja v obliki logike prvega reda, s
pomočjo katere preverjamo novo vnešeno znanje, pretvornike iz logike v naravni jezik ter trenutni kontekst vsakega uporabnika, s pomočjo katerega sistem lahko
iz množice izbere prave uporabnike ob za njih pravem času. Novo pridobljeno znanje je vključeno v obstoječo bazo, s čimer se ustvari pozitivna povratna zanka, kjer obstoječe znanje in kvaliteta novega znanja vplivata en na drugega in se nenehno izboljšujeta skozi čas. Predlagani pristop se lahko vključi v sodobno interakcijo med računalnikom in človekom, ki poteka v naravnem jeziku,
pri čemer se zajemanje znanja opravlja kot stranski učinek primarne interakcije.

Kot del te teze predstavljamo tudi konkretno implementacijo, ki je bila na voljo kot mobilna aplikacija skozi več let in uporabljena s strani več tisoč uporabnikov. Med preizkusom je predstavljen pristop do zbiranja strukturiranega znanja zbral več kot 57.978 odgovorov na vprašanja v naravnem jeziku in iz odgovorov pridobil 386.980 novih koščkov znanja oz. logičnih dejstev. Analiza pridobljenega znanja kaže, da so zbrana dejstva resnična in koristna v 95 \%. Prav tako je uporabniški kontekst za ciljano pridobivanje povečal angažiranost in učinkovitost (število novih trditev na dan na uporabnika se je povečalo za 175 \%).