%-------------------------------------------------------------------------------
% 
\chapter{Symbols}
\label{chapter:symbols}
%-------------------------------------------------------------------------------
%
% A command that adjusts the vertical positioning in the abbreviations and 
% symbols chapters
\chapteradjust
% Correct the width of columns (10pt and 390pt) to fit your needs (they 
% should sum up to 400pt).
% Use \cr instead of \\ to break lines.
\begin{longtable}{@{}p{17pt}@{\hspace{2pt} \dots \hspace{5pt}}p{383pt}@{}}

$\in$ & Element of. Stating $x \in S$ means that $x$ is an element of the 
set $S$. \cr

$\land$ & logical conjunction (and). The statement $A \land B$ is true if both
$A$ and $B$ are true, otherwise it is false.  \cr

$\lor$ & logical disjunction (or). The statement $A \lor B$ is true if $A$ or
$B$, or both true. If both are false, the statement is false.  \cr

$\forall$ & Universal Quantifier (for all). One of the quantifiers used to be
able to convert atomic formulas into propositions. For example, 
$\forall x \in S:P(x)$  means that the propositional function $P(x)$ is true
for every $x$ in the set $S$. Or shorter. $\forall x:P(x)$, means this is true
in the universal set. Given yet another example, non propositional atomic
formula $x > 5$ which is neither true or false, can be converted into true/false
proposition by adding a quantifier: $\forall x:x>5$.\cr

$\exists$ & Existential Quantifier (there exists). One of the quantifiers, 
that can be used to convert atomic formulas into propositions. For example,
$\exists x \in S:P(x)$ means that there exists at least one $x$ in the set $S$,
for which the propositional function $P(x)$ is true. Or shorter, 
$\exists x:P(x)$ means that there exists at least one $x$ in the universal set,
for which the propositional function is true. Given yet another, non 
propositional atomic formula $x > 5$ is neither true or false. But when 
converted to proposition by adding a quantifier it becomes either true or false 
in the given set: $\exists x:x>4$.\cr

$\implies$ & \emph{Material implication}, also known as 
\emph{Material conditional} or simply \emph{implication} is a logical 
connective that is used to form the statements like $p \implies f$, which can
be read as "if $p$ is true, then also $q$ is true". If $p$ is true and $q$ is
false, then the whole statement $p \implies q$ is false. \cr

$\iff$ & \emph{Material Equivalence} is a bi-conditional logical connective
between two logical statements. In English it can read as "Id, and only if.",
or "the same as". The statement $A \iff B$ is true only if both A and B are 
false, or both A and B are true. \cr

\end{longtable}
