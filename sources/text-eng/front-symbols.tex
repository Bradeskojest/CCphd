%-------------------------------------------------------------------------------
% 
\chapter{Symbols}
%-------------------------------------------------------------------------------
%
% A command that adjusts the vertical positioning in the abbreviations and 
% symbols chapters
\chapteradjust
% Correct the width of columns (10pt and 390pt) to fit your needs (they 
% should sum up to 400pt).
% Use \cr instead of \\ to break lines.
\begin{longtable}{@{}p{10pt}@{\hspace{2pt} \dots \hspace{5pt}}p{390pt}@{}}

$\in$ & Element of. Stating $x \in S$ means that $x$ is an element of the 
set $S$. \cr

$\land$ & logical conjunction (and). The statement $A \land B$ is true if both
$A$ and $B$ are true, otherwise it is false.  \cr

$\forall$ & universal quantifier (for all). One of the quantifiers used to be
able to convert atomic formulas into propositions. For example, 
$\forall x \in S:P(x)$  means that the propositional function $P(x)$ is true
for every $x$ in the set $S$. Or shorter. $\forall x:P(x)$, means this is true
in the universal set. Given yet another example, non propositional atomic
formula $x > 5$ which is neither true or false, can be converted into true/false
proposition by adding a quantifier: $\forall x:x>5$.\cr
\end{longtable}
