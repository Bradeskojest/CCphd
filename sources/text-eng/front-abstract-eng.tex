%-------------------------------------------------------------------------------
% 
\chapter*{Abstract}
\pdfbookmark[0]{Abstract}{Abstract}
%-------------------------------------------------------------------------------
Acquisition of high quality structured knowledge has been one of the 
longstanding goals of the Artificial Intelligence (AI) research, including for 
the reason that having such structured knowledge of a high quality, helps to 
advance other research activities from the AI field. With the recent advances 
in crowdsourcing, the sheer number of internet users and the commercial 
availability of crowdsourcing and knowledge acquisition platforms have come a 
new set of tools to tackle this problem. Although numerous systems and methods 
for crowdsourced knowledge acquisition had been developed and solve the problem
of manpower, the issues of high financial costs, task preparation, finding 
the right crowd, consistency, and quality of the acquired knowledge, seem to 
persist. 

This thesis address this deficit by formalizing and implementing an approach
to lower the costs and increase the quality of acquired knowledge. The approach
exploits an existing knowledge base to drive the crowdsourced KA process, 
user contex to communicate with the right people, and again the existing
knowledge, to check for consistency of the user-provided answers. 

We tested the 
viability of the approach in experiments using our platform with real users around the world, and an existing large source of common sense background knowledge. These experiments show that the approach is promising: the knowledge is estimated to be true and useful for users 95\% of the time. Using context to proactively drive knowledge acquisition increased engagement and effectiveness (the number of new assertions/day/user increased for 175\%). Using pre-existing and newly acquired knowledge also proved beneficial.
The
approach applies context aware knowledge acquisition that simultaneously satisfies users immediate information needs while extending its own knowledge using crowdsourcing. The focus is on knowledge acquisition on a mobile device, which makes the approach practical and scalable; in this context, 


which can be incorporated into modern natural language
human-computer Interaction(HCI) iterfaces doing the KA tasks in the background
as part of its standard interaction (primary goal) operations. 



ng an existing knowledge base to drive the acquisition process, address the right people, and check their answers for consistency. We conducted tests of the viability of the approach in experiments with real users, a working platform and common sense knowledge
