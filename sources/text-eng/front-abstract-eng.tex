%-------------------------------------------------------------------------------
% 
\chapter*{Abstract}
\pdfbookmark[0]{Abstract}{Abstract}
%-------------------------------------------------------------------------------
Acquisition of high quality structured knowledge has been one of the 
longstanding goals of the Artificial Intelligence (AI) research, including for 
the reason that having such structured knowledge of a high quality, helps to 
advance other research activities from the AI field. With the recent advances 
in crowdsourcing, the sheer number of internet users and the commercial 
availability of crowdsourcing and knowledge acquisition platforms have come a 
new set of tools to tackle this problem. Although numerous systems and methods 
for crowdsourced knowledge acquisition had been developed and solve the problem
of manpower, the issues of high financial costs, task preparation, finding 
the right crowd, consistency, and quality of the acquired knowledge, seem to 
persist. 

This thesis address this deficit by formalizing and implementing an approach
to lower the costs and increase the quality of acquired knowledge. The approach
exploits an existing knowledge base to drive the crowdsourced KA process, 
user context to communicate with the right people, and again the existing
knowledge, to check for consistency of the user-provided answers. 
The proposed approach can be incorporated into modern natural language
human-computer Interaction(HCI) iterfaces doing the KA tasks in the background
as part of its standard interaction (primary goal) operations, making the KA
process a side-goal while not hindering the main interaction. 

We also implemented the proposed approach as part of the publicly available 
mobile assistant application, acting as a client of proposed KA platform and 
supporting context acquisition algorithms.

We tested the viability of the approach in experiments through multiple years,
using our platform with real users around the world, and an existing large 
source of common sense background knowledge. These experiments show that the 
approach is promising, allowing some previously not possible approach angles 
toward crowd-sourced knowledge acquisition and yealding a high quality of the 
knowledge, as a side effect of trying to act as an assistant and a companion 
to its users.
