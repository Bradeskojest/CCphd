%-------------------------------------------------------------------------------
%
\chapter{Background and Problem Definition}
\label{chapter:background}
%-------------------------------------------------------------------------------
This chapter describes the challenges and components that Knowledge Acquisition 
system such is presented in this work (\emph{Curious Cat}) have to address and 
bring together into an interconnected workflow in order to be a coherent 
KA system able to satisfy the goals of general (common) Knowledge Acquisition. 

\emph{Curious Cat} is a KA system making use of existing knowledge, 
logical inference, crowd-sourcing and mobile context, to trigger natural language
questions at user-appropriate moments and then incorporate the answers consistently
into the existing KB. To successfully do this, there are many inter connected
steps addressing a broad range of as yet not completely solved problems from
multiple fields of artificial intelligence, machine learning, natural language
processing and human computer interaction. Additionally to that, given
that the approach uses crowd-sourcing, there is an additional complexity
of technical implementation and scalability. 

For more structured explanation of the approach and challenges involved, this
section is grouped into sub-sections describing main challenges. Each section
then references our approach to it (implementation) and also related works that
gives overview of similar approaches.

\emph{Curiois Cat} is knowledge driven, meaning that knowledge is connecting
all of the components, including the user interaction, and storing of the
results into the KB (section \ref{section:bg:knowledge}). Its user context is 
obtained through a mobile sensor mining in a real world application that 
monitors the user’s activity and location through mobile GPS and accelerometer 
sensors. This raw data is then corrected, clustered, classified and enriched 
before inserted into the KB as knowledge (section \ref{section:bg:context}.
The newly asserted context can trigger forward chaining operation of the 
inference engine (section \ref{section:bg:knowledge}) which can results in
logical representation of a new question (section \ref{section:bg:ka}) or a 
statement that the system intends to show to the user. The aforementioned  
logical formula is then converted to natural language (sections 
\ref{section:bg:nlp}) and presented to the user through a mobile app. 
When the user answers, his NL answer is converted back to logic 
(section \ref{section:bg:nlp}), checked by the inference engine agains the 
existing knowledge for consistency, and inserted as new peace of knowledge into 
the KB (section \ref{section:bg:ka}). After iteraction like that, the
system determines whether to continue the conversational path with the user or
 not. Newly acquired knowledge is then used to check with other users for
validity (section \ref{section:bg:crowdsourcing}) and is then used by the 
inference to produce new questions/comments/suggestions (section 
\ref{section:bg:ka}).

\section{Knowledge Representation, Engineering and Inference}
\label{section:bg:knowledge}
The proposed Knowledge Acquisition (KA) approach described in 
\autoref{chapter:approach} is completely knowledge driven. One of the main 
building blocks of our system is its knowledge base, which must be based on a 
representation language expressive enough to support the knowledge structures 
required to drive the behavior of the system, and to represent the wide variety 
of knowledge it may gather from the users. Additionally, the inference engine 
needs to be able to access knowledge from the KB and needs to be powerful enough 
to perform inference over it.

%subsection
\subsection{Knowledge Base}
\label{section:bg:kb}
The Knowledge Base (KB) is a crucial part of our KA approach 
(marked in purple in \autoref{fig:Architecture}). It dictates the expressivity 
of the knowledge representation that we must use, as well as 
(together with the inference engine) the overall speed of the KA system. 
Although our approach is generally KB independent, each KB has its own 
specific characteristics, which need to be taken into consideration when 
representing and storing the acquired knowledge. As a part of this research we 
have considered three knowledge bases and the appropriateness of their 
knowledge representations.
\begin{enumerate}
\item The main system reported here was built based on the Cyc KB
\parencite{Lenat1995}, in a form similar to Research Cyc. 
\item Inspired by the Open Cyc representation we also developed our own 
knowledge base and inference engine, Umko, designed to be as simple as possible
while still supporting several of the typical use-cases presented here. 
Because of its simplicity, the approach built around Umko can run in totality 
on a 2015-quality smart phone.
\item Additionally, to test the generality of the approach, we used similar 
idea with a standard RDF knowledge representation\parencite{Bradesko2012}.
\end{enumerate}

Requirements and specifics about KB are described in more detail in 
\autoref{section:kb} and also in the implementation \autoref{section:cyckb}.

%subsection Inference engine
\subsection{Inference Engine}
\label{section:bg:inference}
In addition to the KB, a second most crucial part of the proposed system is an 
inference engine (marked in red in \autoref{fig:Architecture}) that can reason 
over the knowledge stored in the KB. The inference engine is used to detect 
the missing knowledge and infer what to ask and when, based on the context 
that is asserted as a part of the knowledge. Similarly to the knowledge base, 
the inference engine influences the speed and complexity of the approach. Our 
approach was again twofold. The main experiments were based on the Cyc 
inference engine (see \autoref{section:cycinference}) which is tightly linked 
to the Cyc KB and thus able to apply inference over the biggest existing 
common sense KB. Additionally, we developed the inference capabilities 
(forward and backward chaining) inside our custom developed inference 
engine - Umko, needed to test the generalizability of the proposed approach 
and to make it possible to run on embedded devices.

%subsection
\subsection{Knowledge Representation}
\label{section:bg:representation}
The problem of representing knowledge in a computer is as old as the field of 
the AI research and has been tackled from various perspectives, but still not 
completely solved. The proposed approach relies on the Knowledge Representation
(KR) which is powerful 
enough to describe the real world we live in. Additionally, it needs to cover 
knowledge about the knowledge itself (meta-knowledge), enabling inference over 
internal knowledge structures, questions and answers (see \autoref{section:kakb}
for approach and \autoref{section:cyckb} for the implementation). 
This inference is used to 
produce statements and questions and interactions between them as part of a 
dialogue used to communicate with the user in the process of knowledge 
acquisition. We have tested two approaches, where one was fully based on 
the Cyc KB, and the other based on Umko, where we only created the 
minimal upper ontology and vocabulary to support the KA task. 

\section{Context (Information) Extraction}
\label{section:bg:context}
To be able to ask relevant questions, connected to the user or something that 
the user is currently doing or knows, and to ask at the right time/place, 
maintaining and using user context is crucial. Using context and the newly 
acquired knowledge to drive the KA process, is one of the main contributions 
of this paper. 

Nowadays, the obvious information source to get to know the user is his or her
mobile phone, which, with its sensors can provide an extensive and valuable 
source of information. However, the users need to approve using their phones 
sensors, and additionally, the raw measurements of sensor data need to be 
processed and understood to the extent that it is possible to use that 
information in some sort of knowledge representation. In our approach, we used 
two types of the context. One was provided by the users directly via answering 
questions about themselves (\autoref{section:bg:contextu} below and 
\autoref{section:acquiredContext}). Another part 
of the context was mined from the phone sensors as introduced in 
\autoref{section:bg:contextm} and described in sections 
\ref{section:minedContext} and \ref{section:locationServlet}.


\section{Knowledge Acquisition}
\label{section:bg:ka}
dada

\section{Natural Language Processing}
\label{section:bg:nlp}
dada

\section{Crowdsourcing}
\label{section:bg:crowdsourcing}
dada
