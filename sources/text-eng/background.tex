%-------------------------------------------------------------------------------
\chapter{Background and Problem Definition}
\label{chapter:background}
This chapter describes the challenges and components that Knowledge Acquisition 
system such is presented in this work (\emph{Curious Cat}) have to address and 
bring together into a working workflow in order to be a coherent KA system able 
to satisfy the goals of general (common) Knowledge Acquisition. 

\emph{Curious Cat} is a KA system making use of existing knowledge, 
logical inference, crowd-sourcing and mobile context, to trigger natural language
questions at user-appropriate moments and then incorporate the answers consistently
into the existing KB. To successfully do this, there are many inter connected
steps addressing a broad range of as yet not completely solved problems from
multiple fields of artificial intelligence, machine learning, natural language
processing and human computer interaction. Additionally to that, given
that the approach uses crowd-sourcing, there is an additional complexity
of technical implementation and scalability. 

For more structured explanation of the approach and challenges involved, this
section is grouped into sub-sections describing main challenges. Each section
then references our approach to it (implementation) and also related works that
gives overview of similar approaches.

\emph{Curiois Cat} is knowledge driven, meaning that knowledge is connecting
all of the components, including the user interaction, and storing of the
results into the KB (section \ref{section:bg:knowledge}). Its user context is 
obtained through a mobile sensor mining in a real world application that 
monitors the user’s activity and location through mobile GPS and accelerometer 
sensors. This raw data is then corrected, clustered, classified and enriched 
before inserted into the KB as knowledge (section \ref{section:bg:context}.
The newly asserted context can trigger forward chaining operation of the 
inference engine (section \ref{section:bg:knowledge}) which can results in
logical representation of a new question (section \ref{section:bg:ka}) or a 
statement that the system intends to show to the user. The aforementioned  
logical formula is then converted to natural language (sections 
\ref{section:bg:nlp}) and presented to the user through a mobile app. 
When the user answers, his NL answer is converted back to logic 
(section \ref{section:bg:nlp}), checked by the inference engine agains the 
existing knowledge for consistency, and inserted as new peace of knowledge into 
the KB (section \ref{section:bg:ka}). After iteraction like that, the
system determines whether to continue the conversational path with the user or
 not. Newly acquired knowledge is then used to check with other users for
validity (section \ref{section:bg:crowdsourcing}) and is then used by the 
inference to produce new questions/comments/suggestions (section 
\ref{section:bg:ka}).

\section{Knowledge Representation, Engineering and Inference}
\label{section:bg:knowledge}
dada

\section{Context (Information) Extraction}
\label{section:bg:context}
dada

\section{Knowledge Acquisition}
\label{section:bg:ka}
dada

\section{Natural Language Processing}
\label{section:bg:nlp}
dada

\section{Crowdsourcing}
\label{section:bg:crowdsourcing}
dada
