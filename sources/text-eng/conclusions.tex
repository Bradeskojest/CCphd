%-------------------------------------------------------------------------------
% 
\chapter{Conclusions}
\label{chapter:conclusions}
%-------------------------------------------------------------------------------
This thesis proposes a novel novel mobile, context aware,
conversational crowdsourcing knowledge acquisition approach which is able to 
gather new knowledge of a very high quality in a continuous way, while 
interacting with its users using natural language. This is achieved by using an
existing knowledge base, users current and past context, and employing a 
targeted crowdsourcing methodology. We proposed to use highly focused context 
to support targeting the right users 
at the right time. This had (to our knowledge) not been tried before. The
proposed approach was implemented as part of Curious Cat system and evaluated 
on the the data gathered during the experiment running online throughout four 
years. 

The results of the experiment showed that the approach is feasible and gives a 
good quality results as knowledge that is immediately useful, even if the users 
(crowd) does not have a knowledge engineering background. It also shows that
it is possible to incorporate KA into the natural language HCI interaction and 
thus gather knowledge as a side effect of using the application for some other
purposes. The results also confirm that all of the main features 
of the approach are contributing to the quality and quantity of the collected 
data, while making users interested enough (25\% using the app for more than a 
day, 2\% for more than a year), besides being in a research prototype state. 
The context and pro-activity increase engagement from 7.1 assertions/day to 
42.1/day. Using newly acquired knowledge to produce more and better questions 
acquired 39.4\% of all the knowledge. On the other side consistency checking 
improved the knowledge base by preventing bad assertions (2.41\% of all user 
answers) and crowd voting prevented consistent but otherwise wrong or untrue 
assertions (1.26\% of all user answers).

%section
\section{Scientific Contributions}
As mentioned already in the introduction, and showed through the rest of the
chapters, the work presented in this thesis consists of the following 
contributions toknowledge acquisition and also conversational agents:
\begin{itemize}
	\item Definition of the framework for pre-existing knowledge and context
	driven knowledge acquisition by automatically generating crowd-sourcing tasks in natural language.
	\item Real world implementation of the framework with the experiments 
	that lasted throughout more than 4 years including 2411 authenticated 
	users.
	\item An insight into new paradigm of constructing conversational agents,
	shifting from natural language patterns to first order logic rules as part of knowledge driven conversation construction.
\end{itemize}
%section
\section{Future Work}
There are several potential directions for the future work, depending which 
part of the system we want to improve first. 
To improve the user engagement (besides the obvious one to improve and finish
the system or make it part of some really useful application), is to apply
machine learning on top of using the only inference rules. Since the system
already collects various features for when and how much users answer, we could
try to apply some ranking algorithms to improve the ordering of the questions 
(at the cases when we have multiple of them for the same concept), and thus
either optimize the knowledge gain, or user interest, or both if possible.

The other direction is to extend the system to be able to acquire also the
predicates and the rules, by finding a ways to ask questions which combine 
multiple concepts in a new ways and let them get confirmed by the users. 
Another direction is to analyze the free text questions and answers that users
give, especially through MMHHI (\autoref{section:mmhhi}), and mine the patterns,
rules and predicates from these, then use the described KA approach to verify
and confirm the mined knowledge.

One of the other promising directions for the future work is to improve the
approach to become a full conversational engine (or to be merged with some 
existing one), which would make it easy to construct knowledgeable chat-bots. 
This would allow improvement of the system to accept and understand more 
complex answers and questions from the users, and to not limit itself only to 
short replies and guided responses. This is to some extent similar to the 
current AIML and ChatScript systems, except that \emph{Curious Cat} is 
completely knowledge driven and designed to be able to extend itself 
indefinitely while talking with the users, and is only using patterns for NL 
to knowledge conversion (as opposed to the existing approaches). 

