%-------------------------------------------------------------------------------
% 
\chapter{Conclusions}
\label{chapter:conclusions}
%-------------------------------------------------------------------------------
This thesis proposes a novel mobile, context aware,
conversational crowdsourcing knowledge acquisition approach which is able to 
gather new knowledge of a very high quality in a continuous way, while 
interacting with its users using natural language. This is achieved by using an
existing knowledge base, users current and past context, and employing a 
targeted crowdsourcing methodology. We proposed to use highly focused context 
to support targeting the right users 
at the right time. This had (to our knowledge) not been tried before. The
proposed approach was implemented as part of Curious Cat system and evaluated 
on the data gathered during the experiment running online throughout four 
years. 

The results of the experiment showed that the approach is feasible and gives a 
good quality results as knowledge that is immediately useful, even if the users 
(crowd) do not have a knowledge engineering background. It also shows that
it is possible to incorporate KA into the natural language human computer interaction and 
thus gather knowledge as a side effect of using the application for some other
purposes. The results also confirm that all of the main features 
of the approach are contributing to the quality and quantity of the collected 
data, while making users interested enough (25\% using the app for more than a 
day, 2\% for more than a year), besides being in a research prototype state. 
The context and pro-activity increase engagement from 7.1 assertions/day to 
42.1/day. Using newly acquired knowledge to produce more and better questions 
acquired 39.4\% of all the knowledge. On the other side consistency checking 
improved the knowledge base by preventing bad assertions (2.41\% of all user 
answers) and crowd voting prevented consistent but otherwise wrong or untrue 
assertions (1.26\% of all user answers).

%section
\section{Scientific Contributions}
As mentioned already in the introduction, and showed through the rest of the
chapters, the work presented in this thesis consists of the following 
contributions to knowledge acquisition and also conversational agents:
\begin{enumerate}
	\item \emph{Definition of the framework for preexisting knowledge and
	context	driven knowledge acquisition by automatically generating 
	crowd-sourcing tasks in natural language.} The approach with most of its
	 mechanisms and its feasibility are presented in chapter 
	 \emph{\nameref{chapter:approach} (\autoref{chapter:approach})}. 
	This chapter
	also shows that, while the approach was inspired by \emph{Cyc} system and needs an expressive KB, logic,	NL tools and meta-reasoning capability of
	the inference engine, it is generally applicable and not fixed to
	particular instances of these tools. This is presented through the 
	KB	definitions that we developed from scratch and are also needed to be
	able to explain the approach. The approach is also validated, showing that
	it is possible to use it for acquisition of a high quality knowledge. 
	Validation also shows that using the contextual knowledge about the user,
	increases the rate of KA for 175\% (\autoref{tab:ccresultscompare}).
	
	\item \emph{Real world implementation of the framework with the experiments 
	that lasted throughout more than 4 years including 2411 authenticated 
	users.} The proposed approach was implemented to the prototype level as an AI Assistant \emph{Curious Cat}, and was as such one of the first public and larger scale deployment of Cyc system. \emph{Curious Cat} was used by
	 2,401 users (without deliberate marketing), showing that it is possible to
	 do KA as part of some
	other task (acting as an AI assistant). This implementation also proved 
	the feasibility of the approach acquiring high quality knowledge, and 
	also other valuable data about how such a system can be used for KA. Besides
	the AI assistant implementation, there were two more implementations, 
	emphasizing the generality of the approach (Please refer to \autoref{chapter:implementation}).

	\item \emph{An insight into new paradigm of constructing conversational agents,	shifting from natural language patterns to first order logic rules as part of knowledge driven conversation construction.} 
	The side effect of KA process with its arrangement of the acquisition tasks into topics and follow up reposes to user answers, is that it gives an appearance of  intelligent	chat-bot. That it is perceived as one, was also visible from some user's answers, where instead of answering questions they were responding with "chatty" responses like "Hi, I am X. What is your name?".
	As described in this work, the main difference and improvement compared to existing chat-bot technologies is, that in our approach, a big preexisting logical KB with	inference rules replaces linguistic patterns of stimuli and response combinations. Also, at the same time as doing conversation, 
	our system does the KA all the time. By using the preexisting KB we were able to	filter out almost all of wrong answers that our system couldn't
	properly parse. For 143 days
	of data we used for the evaluation, our system rejected 563 answers out of 22,980 given (\autoref{tab:resultsconsistencycheck}), keeping only logically valid. Later manual analysis (\autoref{tab:validity}) showed that a out of 100 of valid answers, almost all were also true (96).
	
	On the other side, using	a big preexisting KB (CycKB) covers enough aspects of common sense knowledge for the system to be able to discuss about quite a lot of topics and give some feeling of understanding to the users. Since the patterns that our system
	can parse are converted to the logic, where are semantically connected 
	in a meaningful way, consequently the responses it can generate reflect a "real understanding". While this approach has the same problems as a standard chat-bots, that it needs a lot of patterns to be able to respond
	in a meaningful way, it is still an improvement, since when the pattern is
	parsed, it guarantees that the response will be logically valid. Regardless of the fact that the chat bot aspect of our approach was a side effect, it
	is interesting enough that we think it is worth exploring this direction in some more detail in the future research. 
\end{enumerate}

%section
\section{Future Work}
There are several potential directions for the future work, depending which 
part of the system we want to improve first. 
To improve the user engagement (besides the obvious one to improve and finish
the system or make it part of some really useful application), is to apply
machine learning on top of using the only inference rules. Since the system
already collects various features for when and how much users answer, we could
try to apply some ranking algorithms to improve the ordering of the questions 
(at the cases when we have multiple of them for the same concept), and thus
either optimize the knowledge gain, or user interest, or both if possible.

The other direction is to extend the system to be able to acquire also the
predicates and the rules, by finding a ways to ask questions which combine 
multiple concepts in a new ways and let them get confirmed by the users. 
Another direction is to analyze the free text questions and answers that users
give, especially through MMHHI (\autoref{section:mmhhi}), and mine the patterns,
rules and predicates from these, then use the described KA approach to verify
and confirm the mined knowledge.

One of the other promising directions for the future work is to improve the
approach to become a full conversational engine (or to be merged with some 
existing one), which would make it easy to construct knowledgeable chat-bots. 
This would allow improvement of the system to accept and understand more 
complex answers and questions from the users, and to not limit itself only to 
short replies and guided responses. This is to some extent similar to the 
current AIML and ChatScript systems, except that \emph{Curious Cat} is 
completely knowledge driven and designed to be able to extend itself 
indefinitely while talking with the users, and is only using patterns for NL 
to knowledge conversion (as opposed to the existing approaches). 

