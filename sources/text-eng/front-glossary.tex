%-------------------------------------------------------------------------------
% 
\chapter{Glossary}
%-------------------------------------------------------------------------------
\emph{Antecedent (predicate logic)} is a first half of the hypotetical 
proposition. It is a $p$ part of the implies statement (see symbol $\implies$ in
Chapter Symbols for explanation. In an implication $p \implies q$, $p$ is an
antecedent.\\

\emph{Arity (predicate logic)} is a property of predicate that defines the 
number of parameters or operands that predicate can operate with. For example,
if a predicate $P$ has \emph{arity} of 2, valid statements using this predicate
can only be the ones with exactly 2 operands ($P(x,y), P(a,b),...$). Statement
$P(x)$ is in this case not a valid statement, since it uses the predicate with
only one parameter.\\

\emph{AST (Abstract Syntax Tree)} is an abstract representation of Wikipedia 
page as parsed from DBPedia parser. Something like DOM tree for Wikipedia 
instead for pure HTML\\

\emph{Atomic Formula} (predicate logic). If the predicate $P$ has arity $n$,
then $P$ followed by $n$ constants and variables is an atomic formula. Examples:
$P(a), P(x), P(x,y) D(a,x)$.\\

\emph{Consequent (predicate logic)} is a second half of the hypotetical 
proposition. It is a $q$ part of the implies statement (see symbol $\implies$ in
Chapter Symbols for explanation. In an implication $p \implies q$, $q$ is a
consequent or apodosis.\\

\emph{Constant} (predicate logic) is besides \emph{variables}, 
\emph{predicates}, and \emph{quantifiers} one of the atomic parts of the 
\emph{predicate logic} sentences. For example, in a sentence $P(a,x)$, $a$ 
serves as a constant. Constants are usually marked with the letters from the
beginning of the alphabet. In this thesis, also predicate is a constant and
all constants are written either with letters or their $Names$.\\

\emph{Existential Quantifier ($\exists$)}. For explanation see the symbol
$\exists$ in the chapter Symbols.\\

\emph {First order logic} can also be called \emph{Predicate logic}. See this
term for more refined definition\\

\emph{Material Equivalence ($\iff$)}. For explanation see the symbol $\iff$ in the
chapter Symbols.\\

\emph{Material implication ($\implies$)}. For explanation, see $\implies$ in the
Chapter Symbols.\\

\emph{OIE (Open Information Extraction} is a paradigm introduced by Oren Etizoni
in his TextRunner system. The main idea of this paradigm is that the knowledge 
acquisition system is not pre-determined to extract some specific facts, 
patterns, etc, but is open-ended, extracting large set of relational tuples 
without any human input.\\

\emph{PMI (Pointwise Mutual Information} is a measure which captures 
co-occurence relationsip between terms in a big corpus.\\

\emph{predicate} is a a term used in predicate logic, representing a verb
template that desribes properties of objects, or relationships between multiple
objects.\\

\emph{Predicate logic}, called also \emph{First order logic} is a formal system
that uses quantification over variables. This makes this logic more expressive
than the \emph{Propositional logic}. In some limited sense, 
\emph{Predicate logic} could be defined as \emph{Propositional logic} with 
quantifiers.\\

\emph{proposition}. This term is often synonim for a logical \emph{statement},
but can also mean more abstract meaning that two different statements with the
same meaning represent. In \emph{Propositional logic}, a proposition is the
smalles syntactic unit. On the other hand, in \emph{Predicate logic}, 
statements/sentences are broken into \emph{constatants, variables, predicates}
and \emph{quantifiers}.\\

\emph{Propositional function}, is an atomic function in from the 
\emph{Predicate logic} which is open ended (missing quantifiers) and thus
cannot count as proposition. For example, $P(x)$ is a propositional function,
while $\forall x P(x)$ is a proposition.\\

\emph{Propositional logic}, also known as \emph{sentential} or 
\emph{statement logic}, is the branch of logic that operates with entire
propositions/statements/sentences to form more complicated 
propositions/statements/sentences, and also logical relationships and properties
derived from combining  or altering this statements.\\

\emph{Quantifier} (logical). Quantifiers in \emph{Predicate logic} convert
propositional functions (open ended) into proper propositions which can be true
or false. For example, $P(x)$ is a propositional function, which can get
converted into proper proposition using one of the quantifiers: 
$\forall x P(x)$. For more info look for the terms \emph{Universal Quantifiier}
and \emph{Extistential Quantifier}.\\

\emph{Sentential logic}. See the term \emph{Propositional logic}.\\

\emph{SKSI (Semantic Knowledge Source Integration)} is a \emph{Cyc} sub-system
for external knowledge integration.\\

\emph{Statement logic}. Synonim for \emph{Propositional logic}. For description
see the glossary for this term.\\

\emph{Upper Ontology} (also top-level, foundation or core ontology) is the part
of ontology (or knowledge base), which defines the core objects that serve as a
main knowledge building blocks to construct the full knowledge base.\\

\emph{Universal Quantifier ($\forall$)}. For explanation check the symbol 
$\forall$ in the chapter Symbols. \\
