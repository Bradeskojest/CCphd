%-------------------------------------------------------------------------------
\chapter{Related Work}

In this chapter we will give an overview of approaches and related works on
broader knowledge acquisition research field, information extraction, 
crowdsourcing and geo-spatial context mining. 

Knowledge Acquisition has been addressed from different perspectives by many 
researchers in Artificial Intelligence over decades, starting already in 1970 
as a sub-discipline of AI research, and since then resulting in a big number of 
types and implementations of approaches and technologies/algorithms. The 
difficulty of acquiring and maintaining the knowledge was soon noticed and was 
Goined as \emph{Knowledge Acquisition Bottleneck} in 
1977\parencite{Feigenbaum1977}. In more recent survey of KA approaches 
\parencite{Zang2013}, authors categorize all of the KA approaches into four main
groups, regarding the source of the data and the way knowledge is acquired:
\begin{itemize}
	\item \emph{Labour Acquisition.} This approach uses human minds as the 
    knowledge source. This usually involves human (expert) ontologists manually 
    entering and encoding the knowledge.
	\item \emph{Interaction Acquisition.} As in Labour Acquisition, the source 
    of the knowledge is coming from humans, but in this case the KA is wrapped 
    in a facilitated interaction with the system, and is sometimes implicit 
    rather than explicit.
	\item \emph{Reasoning Acquisition.} In this approach, new knowledge is 
    automaticaGly inferred from the existing knowledge using logical rules and 
    machine inference.
	\item \emph{Mining Acquisition.} In this approach, the knowledge is 
    extracted from some large textual corpus or corpora.
\end{itemize}

We believe this categorization most accurately reflects the current state of 
machine (computer) based knowledge acquisition, and we decided to use the same 
classification when structuring our related work, focusing more on closely 
related approaches and extending where necessary. According to this 
classification, our work presented in this thesis, fits into a hybrid approach 
combining all four groups, with main focus on interaction and reasoning. We 
address the problem by combining the labour and interaction acquisition (users 
answering questions as part of NL interaction aimed at some higher level goal, 
such as helping the user with various tasks), adding unique features of using 
user context and existing knowledge in combination with reasoning to produce a 
practically unlimited number of potential interaction acquisition tasks, going 
into the field of crowd-sourcing by sending these generated tasks to many users 
simultaneously.

\todo{Fix this, reference to chapters instead to specifici works.} 
Previous works that can compare with our solution is divided into the systems 
that exploit existing knowledge (generated anew during acquisition or 
pre-existing from before in other sources) \parencite{Singh2002a,Witbrock2003,
Forbus2007,Kvo2010,Sharma2010,Mitchel2015}, reasoning \parencite{Witbrock2003,
Speer2007,Speer2008,Kuo2010}, crowdsourcing \parencite{Singh2002,Speer2009, 
Kuo2010, Pedro2012a, Pedro2013}, acquisition through interaction 
\parencite{Speer2009,Pedro2012,Pedro2013}, acquisition through labour(\hl{add, 
probably rather refer to subsections}) \parencite{} and natural language 
conversation\parencite{Pedro2012, Speer2007,Speer2009, Witbrock2003,Kuo2010}.

\hl{Test referencing table} (see \tablename~\ref{tab:related}).

\begin{landscape}
	\begin{table}[htb]
	\caption{Structured overview of related KA systems}
	\label{tab:related}
	\centering
	\begin{adjustbox}{width=1.4\textwidth}
	\begin{tabular}{lclcccccc}
		\hline
		System & Parent & Reference & Category & Source & Representation & Prior K. &  Crowds. & Context \\
		\hline
		Cyc project (Cycorp) & / & \parencite{Lenat1995} & Labour & K. Exp. & CycL & / & / & / \\
		ThoughtTrasure(Signiform) & / & \parencite{Mueller2003} & Labour & K. Exp. & LAGS & / & / & / \\
		HowNet (Keen.) & / & \parencite{Dong2010} & Labour & K. Exp. & KDML & / & / & / \\
		OMCS/ConceptNet (MIT) & / & \parencite{Singh2002a} & Labour & Public & ConceptNet & / & \checkmark & / \\
		GAC/Mindpixel(McKinstry) & / & \parencite{McKinstry2008} & Labour & Public & MindPixel & / & \checkmark & / \\
	    SKSI (Cycorp) & Cyc & \parencite{Masters2007} & Labour/Integration & K. Exp. &  Structured & \checkmark & / & / \\
		KRAKEN (Cycorp) & Cyc & \parencite{Panton2002a} & Interaction & D. Exp & CycL & \checkmark & / & / \\
		UIA (Cycorp) & Cyc & \parencite{Witbrock2003UIA} & Interaction & D. Exp & CycL & \checkmark & / & / \\
		Factivore (Cycorp) & Cyc & \parencite{Witbrock2005} & Interaction & D. Exp & CycL & \checkmark & / & / \\
		Predicate Populator (Cycorp) & Cyc & \parencite{Witbrock2005} & Interaction & D. Exp & CycL & \checkmark & / & / \\
		CURE (Cycorp) & Cyc & \parencite{Witbrock2010} & Interaction & D. Exp & CycL & \checkmark & / & / \\
		OMCommons (MIT) & OMCS & \parencite{Speer2007} & Interaction & Public & ConceptNet & \checkmark & \checkmark & / \\
		Freebase (Metaweb/Google) & / & \parencite{Bollacker2008} & Interaction & Public & RDF & / & / & / \\
		20 Questions (MIT) & OMCS & \parencite{Speer2009} & Game & Public & ConceptNet & / & / & / \\
		Verbosity (CMU) &   & \parencite{VonAhn2006a}  & Game & Public & /  & /  & \checkmark  & /  \\
		Rapport (NTU) & ConceptNet &  \parencite{Kuo2009}  & Game & Public  & ConceptNet  & /  & \checkmark  & /  \\
		Virtual Pet (NTU) & ConceptNet  &  \parencite{Kuo2009}  & Game &  Public & ConceptNet  & /  & \checkmark  & /  \\
		GOKC (NTU) & ConceptNet  & \parencite{Kuo2010}  & Game & Public  & ConceptNet  & \checkmark  & \checkmark  & /  \\
		Collabio (MS) & /  & \parencite{Bernstein2010}  & Game & Public  & /  & /  & \checkmark  & /  \\
		AIML (Alice foundation) & /  & \parencite{Wallace2003}  & Chatbot & /  & AIML  & /  & /  & /  \\
		Chatscript (Brilligunderstanding) & /  & \parencite{Wilcox2011}  & Chatbot & /  & ChatScript  & /  & /  & /  \\
		CyN (Daxtron Labs) & Cyc+AIML  & \parencite{Wilcox2011}  & Chatbot & /  & AIML+Cyc  & \checkmark  & /  & /  \\
	    PCW (Cycorp) & Cyc  & \parencite{Matuszek2004}  & Mining & Web Search  & AIML+Cyc  & \checkmark  & /  & /  \\
 		Learning Reader (NU) & Cyc & \parencite{Forbus2007} & Mining & Web & CycL & \checkmark & / & / \\
		NELL (CMU) & / & \parencite{Mitchell2015} & Mining & Web & Predicate l. & \checkmark & \checkmark & / \\ 
		KnowItAll(UW) & / & \parencite{Etzioni2004} & Mining & Web Search & text & / & / & / \\
		Probase (MSR) & / & \parencite{Wu2012}  & Mining & Web & Propriatery & / & / & / \\
        TextRunner (UW) & KnowItAll & \parencite{Soderland2007} & Mining & Web & text & / & / & / \\
		ReVerb (UW) & TextRunner & \parencite{Fader2011} & Mining & Web & text & / & / & / \\
		R2A2 (UW) & ReVerb & \parencite{Etzioni2011} & Mining & Web & text & / & / & / \\
		ConceptMiner (MIT) & ConceptNet & \parencite{Eslick2006} &Mining & Web Search & ConceptNet & \checkmark & / & / \\ 
		DBPedia (UL\&UoM) & Wikipedia & \parencite{Lehmann2015} & Mining & Wikipedia &  RDF & / &\checkmark & / \\
		YAGO (MPI) & Wikipedia & \parencite{Suchanek2008} & Mining & Wikipedia &  RDF & \checkmark & / & / \\
		Cyc+Wiki (TUW) & Cyc/Wikipedia & \parencite{Medelyan2008} & Mining & Wikipedia &  CycL & / &\checkmark & / \\
		KNEXT (MPI) & \ & \parencite{Schubert2002} & Mining & Penn Treebank &  / & / & / & / \\
	    P. Populator+FOIL (Cyc) & Predicate Populator & \parencite{Witbrock2005} & Reasoning & Induction &  CycL & \checkmark & / & / \\
	    PIP (NU) & Cyc & \parencite{Sharma2010} & Reasoning & Induction &  CycL & \checkmark & / & / \\
	    AnalogySpace (MIT) & OMCS & \parencite{Speer2008} & Reasoning & Analogy &  ConceptNet & \checkmark & / & / \\
		\hline
	\end{tabular}
	\end{adjustbox}
\end{table}
\end{landscape}

\section{Labour Acquisition}
\label{section:LabourAcquisition}
This category consists of KA approaches which rely on explicit human work to 
collect the knowledge. A number of expert (or also untrained) ontologists or 
knowledge engineers is employed to codify the knowledge by hand into the given 
knowledge representation (formal language). Labour acquisition is the most 
expensive acquisition type, but it gives a high quality knowledge. It is often a
crucial initial step in other KA types as well, since it can help to have some 
pre-existing knowledge to be able to check the consistency of the newly acquired
knowledge. Labour Acquisition is often present in other KA types, even if not 
explicitly mentioned, since it is implicitly done when defining internal 
workings and structures of other KA processes. While we checked other well 
known systems that are result of Labour Acquisition, Cyc (mentioned below) is 
the most comprehensive of them and was picked as a starting point and main 
background knowledge and implementation base for this work.

\subsection{Cyc} The most famous and also most comprehensive and expensive 
knowledge acquired this way, is Cyc KB, which is part of Cyc AI system 
\parencite{Lenat1995}. It started in 1984 as a research project, with a premise 
that in order to be able to think like humans do, the computer needs to have 
knowledge about the world and the language like humans do, and there is no other
way than to teach them, one concept at a time, by hand. Since 1994, the project 
continued through Cycorp Inc. company, which is still continuing the effort. 
Through the years Cyc Inc. employed computer scientists, knowledge engineers, 
philosophers, ontologists, linguists and domain experts, to codify the knowledge
in the formal higher order logic language CycL \parencite{Matuszek2006a}. As of 
2006 \parencite{Matuszek2006}, the effort of making Cyc was 900 non-crowdsourced 
human years which resulted in 7 million assertions connecting 500,000 terms and 
17,000 predicates/relations \parencite{Zang2013}, structured into consistent 
sub-theories (Microtheories) and connected to the Cyc Inference engine and 
Natural Language generation. Since the implemtentation of our approach is based
on Cyc, we give a more detailed description of the KB and its connected systems
in \autoref{section:Cyc} on page \pageref{section:Cyc}. Cyc Project is still 
work in progress and continues to live and expand through various research and
commercial projects.

\subsection{ThoughtTreasure} Approximately at the same time(1994) as Cyc Inc. 
company was formed, Eric Mueller started to work on a similar system, which was
inspired by Cyc and is similar in having a combination of common sense knowledge
concepts connected to their natural language presentations. The main 
differentiator from Cyc is, that it tries to use simpler representation compared
to first-order logic as is used in Cyc. Additionally, some parts of 
\emph{ThoughtTreasure} knowledge can be presented also with finite automata, 
grids and scripts\parencite{Mueller1999,Mueller2003}. In 2003 the knowledge of
this system consisted of 25,000 concepts and 50,000 assertions. ThoughtTreasure 
was not so successfull as Cyc and ceased all developments in 2000 and was 
open-sourced on Github in 2015. \hl{link as footnote}.

\subsection{HowNet} started in 1999 and is an on-line common-sense knowledge base 
unveiling inter-conceptual relationships and inter-attribute relationships of 
concepts as connoting in lexicons of the Chinese and their English equivalents. 
As of 2010 it had 115,278 concepts annotated with Chinese representation, 
121,262 concepts with English representation, and 662,877 knowledge base records
including other concepts and attributes \parencite{Dong2010}. HowNet knowledge 
is stored in the form of concept relationships and attribute relationships and 
is formally structured in KDML (Knowledge Database Mark-up Language), consisting
of concepts (called semens in KDML) and their semantic roles.
 
\subsection{Open Mind Common Sense (OMCS)} is a crowdsourcing knowledge acquisition 
project that started in 1999 at the MIT Media Lab\parencite{Singh2002a}. 
Together with initial seed and example knowledge, the system was put online with
a knowledge entry interface, so the entry was crowd-sourced and anyone 
interested could enter and codify the knowledge. OMCS supported collecting 
knowledge in multiple languages. It's main difference from the systems described
above (Cyc, HowNet, ThoughtTreasure) is, that it used deliberate crowdsourcing
and that it's knowledge base and representation is not strictly formal logic, 
but rather inter-connected pieces of natural language statements. As of 2013 
\parencite{Zang2013}, OMCS produced second biggest KB after Cyc, consisting of 
English (1,040,067 statements), Chinese (356,277), Portuguese (233,514), 
Korean (14,955), Japanese (14,546), Dutch (5,066), etc. Initial collection was
done by specifying 25 human activities, where each activity got it's own user 
interface for free form natural language entry and also pre-defined patterns 
like "A hammer is for \underline{\hspace{1.5cm}}", where participants can enter
the knowledge. Although OMCS started to build KB from scratch it shares a 
similarity to our CC system in a sense that it is using crowd-sourcing and also
natural language patterns with empty slots to fill in missing parts. OMCS was
later used in many other KA approaches as a prior knowledge, similar way as we 
use Cyc. After a few versions, OMCS was taken from public access and merged with
multiple KBs and KA approaches into an ConceptNet 
KB\footnote{http://conceptnet.io/} \parencite{Speer2016}, which is now (in 2017)
part of Linked Open Data (LOD) and maintained as open-source project.

\subsection{GAC/MindPixel}
Generic Artificial Consciousness was a bold try to make a general AI based on
the premise that human thinking can be simulated with answering binary yes/no
questions or decisions\parencite{McKinstry2018}, where humans have a natural 
bias toward yes answers. With this in mind, McKinstry founded GAC (later renamed
to MindPixel), where internet crowd would answer yes/no questions for shares
in the company which would commercially exploit the GAC AI. Initially, one such
answer which contributed to GAC knowledge base was called MindPixel, but after
a while this became name for the whole system. With this idea, McKinstry's
MindPixel was one of the first crowd-sourced efforts for collaborative KB
building. While MindPixel shut down after McKinsey stopped working on it,
it inspired \emph{OMCS, later OpenMind} and started the crowdsourcing and
crowd based cross validation of the facts. Besides the \emph{OMCS}, a similar
YES/NO crowdsourcing system lives under the name \emph{Weegy}
\footnote{http://www.weegy.com}.

\subsection{Semantic Knowledge Source Integration (SKSI)}
No matter how big the underlying knowledge base is, there will always be some
missing knowledge, that exists somewhere else. While with knowledge acquisition
it is possible to add these missing peaces, sometimes it makes more sense to
keep this knowledge externally and only link it properly so it can be used.
This is especially true for fast changing data such as stock values, weather
forecast information, real-time measurements of traffic/ production line, etc.
In such cases it is beneficial if one can link and address this data in the 
same way as it would have been included in the original KB. This is the goal
of \emph{SKS}\parencite{Masters2007}
In order to connect the external source, a "wapper" knowledge needs to be
defined in \emph{Cyc} KB, which describes which concepts and instances external
source represents, and how to access them (http request, SQL Query, etc.).

In \emph{SKS} system, this "wrapper" knowledge is asserted as normal \emph{CycL}
assertions using special predicates and concepts. The descriptions are 
structured into three layers, where first (access layer) describes 
how to access the data (ie. how to connect, send queries and retrieve the 
content). Then the second (physical schema layer), describes how the data is
structured inside the original source. The third layer (logical schema) 
describes in \emph{CycL}, how  the data connects to \emph{CycKB}. Example
of the logical layer is semantics on how specific columns translate into the KB.
For example, table \emph{Securities}, column \emph{Name}, translates into
instance of Cyc concept \emph{\#\$Equity-Security} with the 
\emph{\#\$nameString} linked to the value of the table cell.
With this approach, it is possible to seamlessly link the existing KB with
external sources, and use \emph{Cyc} inference engine and querying mechanisms
on externally connected data without even noticing it's external.


\section{Interaction Acquisition}
Similarly as with Labour KA, interaction Acquisition gets the knowledge from 
human minds, but in this case the acquisition is an intended side effect, while
users are interacting with the software as part of some other activity/task, or
as part of a motivation scheme, such as knowledge acquisition games. Besides 
games, the interaction could be some other user interface for solving specific
tasks, or a Natural Language Conversation. This type of acquisition is most 
strongly correlated with the approach described in this thesis, since Curious 
Cat uses points (gaming), to motivate users and it interacts with user in NL, 
while discussing various topics (concepts). It uses the conversation to set up
the context and acquire (remember) user's responses and places them properly in
to the KB. Sometimes the acquired knowledge is paraphrased and presented back to
user to show the 'understanding', which was first tried in OSMC (
\autoref{section:LabourAcquisition}, \parencite{Singh2002b}), but there only in
non-conversational way as part of the input forms.
 
\subsection{Interactive User Interfaces}
Interactive user interfaces are the most common representation of interaction 
acquisition, where the user interface is constructed in a way to help user enter
the data and thus make the acquisition much faster and cheaper. Historically, 
these systems were developed to help the labour acquisition systems, or on top
of them, after parent systems reached some sort of maturity and initial 
knowledge stability. This is the reason why all of these systems rely or are 
build on top of labour acquisition (\autoref{section:LabourAcquisition}) or 
mining acquisition (\autoref{section:MiningAcquisition}) systems.

\subsubsection{KRAKEN} system was a knowledge entry tool which allows domain experts to
make meaningful additions to CYC knowledge base, without the training in the 
areas of artificial intelligence, ontology development, or knowledge
representation\parencite{Panton2002a}. It was developed as part of DARPA's
Rapid Knowledge Formation (RKF) project in 2000. As its goal was to allow
knowledge entry to non-trained experts, it started to use natural language 
entry and is as this, a first pre-cursor to Curious Cat system and a seed idea
for it. It consists of creators, selectors, modifiers of Cyc KB building blocks,
tools for consistency checks and tools for using existing knowledge to infer new
things to ask. This tool, together with it's derived solutions was later 
re-written and integrated into Cyc as CURE system (see below). While KRAKEN and
later CURE already used Natural Language generation and parsing, and started 
with the idea of natural language dialogue for doing the KA, the interaction, it
was missing user context (user's had to select or search the concept of 
interest), and also crowdsourcing aspects. Kraken was also missing rules for
explicit question asking. The questions were all related to the selected concept
and given as a list of natural language forms.

\subsubsection{User Interaction Agenda (UIA)} was a web  based user interface for KRAKEN
KA tool\parencite{Panton2002a,Witbrock2003UIA}. It worked inside a browser and 
it worked as responsive web-app (in 2001) by automatically triggering refresh 
functionality of the browser. It consisted of a menu of tools that is organized
according to the recommended steps of the KE process, text entry box (query, 
answer, statement), center screen for the main interaction with the current 
tool, and a summary with a set of colored steps needed to complete current 
interaction. Similarly as KRAKEN itself, this interface was later improved
and integrated into main Cyc system as part of CURE tool. 

\subsubsection{Factivore} 
This application was a Java Applet user interface for an extended KRAKEN system,
meant for quick facts entering \parencite{Witbrock2005}. On the back-end it used
the same mechanisms and logical templates, while in the front-end it only
allowed facts entering, as opposed to UIA, which also allowed rules (which
ended up as not being useful).

\subsubsection{Predicate Populator} 
\emph{Predicete Populator} is a similar tool as \emph{Factivore}, which instead
of only collecting instances, allows to add general knowledge about classes. For
example, instead of describing facts for a specific restaurant, it can collect
general knowledge that is true for all restaurants \parencite{Witbrock2005}. The
context of the KA in this case, is given by class concept, a predicate and a 
web-site which is parsed into CycL concepts. These are then filtered out if they
do not match argument constraints of the predicate and then shown to user for 
selection. As part of the validation, this tool had some problems with correctly
acquired knowledge. One of the proposed solutions (never implemented), was to
start using volunteers to vote about the correctness. This is already a 
pre-cursor idea for crowd-sourced voting mechanisms that we used in Curious Cat.

\subsubsection{Freebase} 
\emph{Freebase} started in 2007\parencite{Bollacker2008} and was a large (mostly
instance based) crowd-sourced graph database for structured general human 
knowledge. Initially it was acquired from multiple public sources, mostly 
Wikipedia. The initial seed was then constantly updated and corrected by the 
community. On the user interface side, Freebase provides an AJAX/Web based 
UI for humans and an HTTP/JSON based API for software access. For finding
knowledge and also software based editing, it uses Metaweb Query Language 
(MQL). A company behind freebase was bought by Google in 2010 and incorporated
into a Google Knowledge Graph. In 2016 Freebase was incorporated into the 
Wikidata platform and shut down by Google and is no longer maintained.

\subsubsection{OMCommons (Open Mind Commons)} 
This system is an interactive interface to OMCS which
can respond with a feedback to user answers and maintain dialogue 
\parencite{Speer2007}. This is similar approach as we do with Curious Cat and
shows understanding of the knowledge users enter. The mechanisms behind is
by using inference engine to make analogical inferences based on the existing 
knowledge and new entry. Then it generates some relevant questions and asks 
user to confirm them. For example, as given from the original paper, 
\emph{OMCommons} asks: "A bicycle would be found on the street. Is this common 
sense?". This is then displayed to the user with the justification for the 
question: "A bicycle is similar to a car. I have been told that a car would 
be found on the street". Users then click on "Yes/No" buttons to confirm or
reject the inferred statement. The interactive interface also allows its users 
to refine the knowledge entered by other users and see the ratings. Users can 
also explore what new inferences are result of their new contributions.

\subsection{Games}
Games are a specific sub-section of interaction acquisition, where the actual
acquisition is hidden or transformed into much more enjoyable process, 
maximizing the entertainment of the users. This type of KA was first 
officially introduced by Luis von Ahn in 2006 \parencite{VonAhn2006,
VonAhn2008} under the name 'Games with Purpose' paradigm.

\subsubsection{20Q (20 Questions)} 
This is a game with intentional knowledge acquisition task
which focuses to the most salient properties of concepts. The game itself is
a standard 20 questions game which aims to make one player figure out the 
concept of discussion by asking yes/no questions and then infer from the 
answers what the concept could be. The only difference is that the player which
is asking is a computer based on OMCS knowledge base. It generates questions in
NL, and according to what a player answers, it attempts to guess the concept.
To decide what questions to ask, it uses statistical classification methods
\parencite{Speer2009}, to discover the most informative attributes of concepts
in OMCS KB. After the user answers all the questions, including whether the
detected concept was right or not, the concept and the answers will be assigned
to proper cluster and thus the characteristics of the object are learned.

\subsubsection{Verbosity}. 
Similarly as Q20 above, Verbosity is a spoken game for two 
persons randomly selected online. It was inspired by Taboo board 
game\parencite{TabooGame} which required players to state common sense facts
without mentioning the secret concept. While having similar game-play as 
aforementioned board game, Verbositywas developed with the intent to collect
common sense knowledge \parencite{VonAhn2006a}. One player (narrator), gets
a secret word concept and needs to give hints about the word to the other 
player (guesser), who must figure out the word that is described the hints.
The hints take the form of sentence templates with blanks to be filled in. 
For example, if the word is "CAR", the narrator could say "it has wheels."
In the experiments, a total of 267 people played the game and collected
7,871 facts. While these facts were mostly a good quality and it was proven
that the game can be used successfully, these facts were natural language 
snippets and were not incorporated into any kind of structure or formal KB. 

\subsubsection{Rapport} 
This is a KA game based on Chinese OMCS questions, but implemented as a 
Facebook game to make use of the social connections inside social network. 
The Game helps users to make new friends or enhance connections with their 
existing social network by asking and answering questions and matching the 
answers to other users\parencite{Kuo2009}. This game aims to enhance the 
experience and community engagement and thus functionality of aforementioned 
\emph{Verbosity} game, by employing simultaneous  interaction between all the 
players versus only 1 to 1 interaction between 2 community members. For 
evaluation, the answers where multiple users answered the same were considered 
valid. This game had a similarity with Curious Cat in a sense that it employed
the voting mechanism for the same answers, and the repetitive questioning of
the same question to multiple users. Authors found out that the agreement 
between same answers of the repetitive question and voting is 80\% or more 
after at least 2 repetitions of the same question. In 6 months, \emph{Rapport}
collected 14,001 unique statements from 1,700 users. Normalized, this is
8.2 answers per user.

\subsubsection{Virtual Pet} 
This is a similar game as \emph{Rapport} in a sense that it
uses \emph{OMCS} patterns, is in Chinese and is developed by the same authors
\parencite{Kuo2009}. Instead of Facebook platform, \emph{Virtual Pet} uses
PTT (Taiwanese bulleting board system in Chinese language). Instead of direct
interaction between the users themselves, users interact with virtual pet and
can ask it questions and answer it's questions. In the back-end, the questions
the pet asks, are actually questions from other users. This game in 6 months 
collected 511,734 unique pieces of knowledge from 6,899 users. Normalized this
is 74,1 answers per user. While this game attracted much more answers than 
\emph{Rapport}, the quality of the answers was slightly lower. Authors
argue that the reasons behind both is, that users didn't interact directly,
but through the virtual pet, so they were less careful whether answers are
correct or not.

\subsubsection{Goal Oriented Knowledge Collection (GOKC)}. 
This game builds on the
findings and approach of \emph{Virtual Pet} KA game. The main improvement
is to try and actually make use of the new knowledge inside a given domain
(picked by the initial seed questions), to infer new questions. With this
the authors tried to fix a drawback of \emph{Virtual Pet}, that through
time, the questions and answers become saturated, and the number of new
questions and answers falls exponentially through time, with respect
to the number of already collected knowledge peaces.
This approach is also aligned with the CC approach, which uses existing+
context and new knowledge, to drive the questions. First part of the \emph{GOKC}
paper describes analysis of the knowledge collected by \emph{Virutal Pet} game. 
The second part is a description and evaluation of  GOKC KA approach, where 
authors did 1 week experiment to show that the approach works. During that 
week the system inferred created 755 new questions, out of which, 12 were
reported as bad. Out of these questions 10,572 answers were collected where
9,734 were voted as good. This results in the 92,07\% precision. Compared
to the game without question expansion (\emph{Virtual Pet}), which has
precision of 80.58\%, this is an improvement.

\subsubsection{Collabio (Collbaorative Biography)}. 
This is also a Facebook based game, with the intention to
collect user's tags. While the gathered knowledge is more a set of person's 
tags than knowledge, it served as an inspiration to \emph{Rapport} and 
\emph{Virtual Pet}. During the experiment, \emph{Collabio} users tagged
3,800 persons with accurate tags with information that cannot be found 
otherwise\parencite{Bernstein2009, Bernstein2010}.

\subsection{Interactive Natural Language Conversation}
Natural Language Knowledge Acquisition methods are special case of Interaction 
Acquisition systems. While almost all of the approaches already described above
(under Interactive User Interfaces and Games subsections) use natural language 
to some extent, the language processing used is based on relatively small amount
of textual patterns, or statements which are not necessary connected into a 
conversation. Common denominator of these systems is that they intentionally try
to acquire knowledge and then use natural language statements to do this. As a 
side effect and as motivation for users, sometimes consequent questions and 
answers give a feeling of conversation. On the other side chat-bots, start with
the intention to maintain an interesting conversation with the users, and have 
to do knowledge acquisition only to remember facts and parts of the past 
conversations to be able to be smart enough, so users do not lose interest.
Starting with Eliza\parencite{Weizenbaum1966}, these systems evolved, mostly 
directed by Turning Tests\parencite{Turing?}, implemented as Loebner 
competitions, trying to pass it\footnote{http://www.loebner.net}. Through the 
measure of these tests\parencite{Bradesko2012}, among a few propriatery 
chat-bots, two technologies evolved (\emph{AIML, ChatScript}) to be general 
enough and can be used for conversational engine (chat-bot) construction and 
also NL knowledge acquisition.

\subsection{AIML(Artificial Intelligence Mark-up Language} is an XML based 
scripting language. It allows developers of chat-bots, to construct a pre-
defined natural language patterns and their responses. These definitions are 
then fed into an AIML engine, which can match user inputs with the patterns and 
figure out what response to write. AIMLs syntax consists mostly of input rules 
(categories) with appropriate output. The pattern must cover the entire input 
and is case insensitive. It is possible to use a wildcard (*) which binds to 
one or more words. The simplest example of AIML pattern with appropriate 
response is presented in \tablename~\ref{tab:aiml_example}. This pattern 
detects user's questions like "Do you have something on the menu?" and responds 
with "We have everything on the menu."

\begin{table}[htb]
\caption{AIML Example}
\label{tab:aiml_example}
\centering
\begin{tabular}{l}
\hline
\lstset{language=XML,breaklines=true}
\begin{lstlisting}
<Category> 
   <pattern> Do you have * on the menu </pattern> 
   <template>
      We have everything on the menu.
   </template> 
</Category>
\end{lstlisting}  \\
\hline
\end{tabular}
\end{table}

AIML allows recursive calls to its own patterns, which allows for some really 
complicated
and powerful patterns, covering many examples of input. Regarding the knowledge 
acquisition,
AIML has an option to store parts of the textual patterns as variables and thus 
store 
information for later. 

\begin{table}[htb]
	\caption{AIML example of saving info to variables}
	\label{tab:aiml_ka1}
	\centering
	\begin{tabular}{l}
		\hline
		\lstset{language=XML,breaklines=true}
		\begin{lstlisting}
<category>
   <pattern>I just ate *</pattern>
   <template>
      Nice choice! <set name = "food"> <star/></set>
   </template>  
</category>  
<category>
   <pattern>I am hungry</pattern>
   <template>
      Eat another <get name = "food"/>?
   </template>  
</category> 
		\end{lstlisting}  \\
		\hline
	\end{tabular}
\end{table}

The AIML example on \tablename~\ref{tab:aiml_ka1} can remember keywords
following "I just ate" pattern, like "I just ate pizza". If user at some point 
later says "I am hungry", the bot is able to respond with "Eat another pizza". 
In a combination with "<that>" tag, which matches previous computer's response, 
AIML can be used to construct specific knowledge acquisition questions 
(\tablename~\ref{tab:aiml_ka2}).  The given example is using AIML 1.0, which
was later improved with AIML2.0\parencite{Wallace2013} which introduced the
\emph{<Learn>} tag, but mechanism stayed mostly the same.

\begin{table}[htb]
	\caption{AIML Example of remembering answers on specific questions}
	\label{tab:aiml_ka2}
	\centering
	\begin{tabular}{l}
		\hline
		\lstset{language=XML,breaklines=true}
		\begin{lstlisting}
<category>
   <pattern>*</pattern>
   <that>What did you order</that>
   <template> 
      Was it good? <set name = "menuItem"><star/></set>
   </template>  
</category>  
		\end{lstlisting}  \\
		\hline
	\end{tabular}
\end{table}

While AIML language with appropriate engine can remember specific facts, the
mechanism is purely keyword based and cannot really count as structured knowledge. Additionally, since it only remembers direct facts, it would be
really hard to construct an acquisition of all types of food for example.
AIML based chatbots were winning Loebner's competitions in the years from 2000
to 2004, but were later outcompeted by Chatscript based bots and propriatery 
solutions.

\subsection {ChatScript} is an NLP expert system consisting of textual patterns 
rules. It was designed by Bruce Wilcox \parencite{Wilcox2011} and besides
patterns it has mechanisms for defining concepts, triple store for facts, 
own inference engine POS tagger and parser. From the measure of how close the system is to pass the Turing Test as measured by Loebner's competitions, \emph{ChatScript} surpassed \emph{AIML},
and is its successor, since both systems are open sourced. It was designed 
purposely to be simpler to use and have more powerful tools for NLP and knowledge acquisition which is integral part chatbot systems. A simple example
from AIML (\tablename~\ref{tab:aiml_example}) can be re-written in much shorter form as ChatScript rule (\tablename~\ref{tab:chatscript_example}).

\begin{table}[htb]
	\caption{Simple ChatScript example}
	\label{tab:chatscript_example}
	\centering
	\begin{tabular}{l}
		\hline
		\lstset{breaklines=true}
		\begin{lstlisting}
?: (do you have * on the menu) We have everything on the menu.
		\end{lstlisting}  \\
		\hline
	\end{tabular}
\end{table}

Similarly the example from \tablename~\ref{tab:aiml_ka1} can be writen as:
\begin{table}[htb]
	\caption{Simple ka (remembering) example}
	\label{tab:chatscript_ka1}
	\centering
	\begin{tabular}{l}
		\hline
		\lstset{breaklines=true}
		\begin{lstlisting}
s: (I just ate _*) Nice choice!
s: (I am hungry) East another _0?
		\end{lstlisting}  \\
		\hline
	\end{tabular}
\end{table}

Similarly, example from \tablename~\ref{tab:aiml_ka2} in ChatScript looks like:

\begin{table}[H]
	\caption{Simple ChatScript question/answer/remember example}
	\label{tab:chatscript_ka2}
	\centering
	\begin{tabular}{l}
		\hline
		\lstset{breaklines=true}
		\begin{lstlisting}
t: What did you order?
	a: (_*) \$menuItem=_0 Was it good?
		\end{lstlisting}  \\
		\hline
	\end{tabular}
\end{table}

While the above examples repeats the functionality of AIML, ChatScript is more 
powerful and can remember facts in the shape of (subject verb object) and act on
them. This is done with using \emph{\^createfact} and \emph{\^findfact} 
functions.


\subsection {CyN} 
\label{section:rw_CyN}
CyN is an AIML interpreter implementation with additional functionality
to be able to access Cyc inference engine and KB for both, storing the knowledge 
and also for querying\parencite{Coursey2004}. This was done by introduction of new AIML tags:
\begin{itemize}
	\item \emph{<cycterm>} Translates an English word/phrase into a Cyc symbol.
	\item \emph{<cycsystem>} Executes a CycL statement and returns the result.
	\item \emph{<cycrandom>} Executes a CycL query and returns one response at random.
	\item \emph{<cycassert>} Asserts a CycL statement.
	\item \emph{<cycretract>} Retracts a CycL statement.
	\item \emph{<cyccondition>} Controls the flow execution in a category template.
	\item \emph{<guard>} Processes a template only if the CycL expression is true
\end{itemize} 

\section{Mining Acquisition}
\label{section:MiningAcquisition}
This category of KA systems try to make use of big text corpus-es 
available online or otherwise on some digital media. Because the core idea of 
writing is to share information, there is a lot of knowledge in the texts that 
can be extracted and converted into a structured knowledge that can later be used 
by computers. Due to wast size and availability of the data on the internet,
mining is most often done on the web resources. Since most of the data format
from these corpuses is text, these techniques are particularly strong in the
using various NLP techniques, which are often combined with existing knowledge
to correct mistakes and check consistency.

\subsection{Populating Cyc from the Web (PCW)}
\label{section:rw_PCW}
Since whole idea of Cyc system is to gain enough knowledge through manual work,
to be able to learn on itself after some point, the Cyc team is looking int into 
other means of knowledge acquisition which can automatize or speed-up the KA process.
One of the approaches is by mining facts from the Internet by issuing appropriate
search engine queries\parencite{Matuszek2004}.

Because Cyc KB is really big, the first step of this approach is to select appropriate
part of the kb (concepts and related queries) which are in the interest of the system.
For initial experiments  a set of 134 binary predicates was selected. These predicates
were then used to scan the KB and find the missing knowledge, which was converted to 
CycL queries. 
These queries were then converted to NL and queried on a web search engine. The results
are then converted back to CycL through NL to logic engine of Cyc. After the conversion
these are converted back to NL and re-searched on the web, to check whether the results
still hold, and then as the last step, the results are checked for consistency 
(whether they can be asserted into Cyc). From the initial 134 predicates, the system
generated 348 queries, 4290 searches. It found 1016 facts, out of which 4 were rejectec
due to inconsistency, 566 rejected by search engine (not same results), 384 were 
already known to Cyc, and finally 61 new consistent facts were detected as valid.
After human review, the findings were that only 32 facts were actually correct.

\subsection{Learning Reader}
\label{section:rw_LR}
Learning Reader\parencite{Forbus2007} is a prototype knowledge mining system
that combines NLP, large KB (CycKB) and analogical inference into an automated 
knowledge extraction system that works on simplified language texts. The system
uses Direct Memory Access Parsing (DMAP\parencite{Martin1986}) to parse text
and convert it into CycL concepts which are then checked by the inference engine
whether they can form correct CycL statements. The prototype consisted of
30,000 NDAP patterns (a quick approximation would be to imagine CyN patterns-
chapter \ref{section:rw_CyN}). After parsing and syntax checks, found CycL 
sentences were checked by the inference within CycKB, whether knowledge is new
or not. Only new logical statements were then asserted. Additional feature of
this prototype system is that it includes question answering mechanism, which 
can be used by evaluators to check what the system learned. This same mechanism
is also used to try to generate new facts (elaborate) and also questions based
on newly acquired knowledge. The experiment on 62 written stories improved the
recall from 10\% to 37\% and kept the accuracy as 99.7\% compared to original 
100\%. By using additional inference and conjecture based inference, the recall
raised to 60\% while accuracy dropped to 90.8\%.

\subsection{Never Ending Language Learner (NELL)}
\label{section:rw_NELL}
NELL\parencite{Mitchell2015} is a text mining KA system running 24/7 with the
goal to extract knowledge, use this knowledge to improve itself and extract
more knowledge. NELL was started in January 2010 and as of 2015 acuired
80 million confidence weighted new beliefs. NELL consists of many different
learning tasks (for different types of knowledge), where each task also consist
of the performance metrics, so the system can asses itself and check if the 
learning task itself is also improving through time. In 2015 it consisted of
2500 learning tasks. Some of learning task examples:
\begin{itemize}
	\item Category Classification
	\item Relation Classification
	\item Entity Resolution
	\item Inference Rules among belief triples
\end{itemize}
After learning tasks, there is a \emph{Coupling Constraings} component, which
combines results of learning tasks. The potentially useful knowledge gets
asserted into the KB as candidates, where the assertions are checked by knowledge
integrator module which integrates the assertions into the KB, or rejects them.

After some initail initial KB had been gathered, the CMU text-mining knowledge 
NELL also started to apply a crowdsourcing approach\parencite{Pedro2012a},
using natural language questions to validate its KB. In a similar
fashion as Curious Cat, NELL can use newly acquired knowledge, to 
formulate new representations and learning tasks. There is, however, a 
distinct difference between the approaches of NELL and Curious Cat. 
NELL uses information extraction to populate its KB from the web, 
then sends the acquired knowledge to Yahoo Answers, or some other Q/A site, 
where the knowledge can be confirmed or rejected. 
By contrast, Curious Cat formulates its questions directly to users 
(and these questions can have many forms, not just facts to validate), 
and only then sends the new knowledge to other users for validation. 
Additionally, Curious Cat is able to use context to target specific users who 
have a very high chance of being able to answer a question.

\subsection{KnowItAll}
\label{section:rw_KnowItAll}
KnowItAll\parencite{Etzioni2004} is a domain independent web fact extraction 
system that s specific search engine queries to find new instances of specific
classes. It starts it extraction with a small seed of class names and NL patterns
like "NP1 such as NP2". The classes and patterns are then used to find instances,
new classes and also new extraction phrases by analyzing the results of the 
web search engines. For example, Googling: "Cities such as *", will return 
a lot of statements with instances of cities. After these cities are extracted,
the names can be used for furhter Googling and by analyzing the phrases in which
these cities appear, new patterns can be found, and so on. As part of the
experiment, KnowItAll ran for 5 days and extracted over 50,000 instances of
cities, states, countries, actors and films.
To asses the correctnes of the extractions, the system can fill-in the instances
into the various patterns and check the hit-count returned by the search engine./
This then compares by the hit-count of the instance itself and uses this to
asses the probability of the instance really belonging to the detected class.
For example: comparing the hit-count of "Ljubljana", "Cities such as Ljubljana"
and "Planets such us Ljubljana", the system can figure out that Ljubljana is
most likely indeed an instance of the class city.

KnowItAll was the first of the systems that inspired an \emph{Open Information
Extraction (Open IE)} paradigm\parencite{Etzioni2011} which resulted in many 
other IE systems such us TextRunner, ReVerb and R2A2. The main ide of this
paradigm is to avoid hand labeled examples and domain specific verbs and nouns
when approaching textual patterns which can lead to open (without specifying the
targets) knowledge extraction on a web scale.

\subsection{Probase}
\label{section:rw_Probase}
Probase is a probabilistic taxonomy of concepts and instances consisting of
2.7 million of concepts extracted from 1.68 billion of web pages
\parencite{Wu2012}. The main difference between Probase and other KBs is that
Probase is probabilistic as opposed of "black and white" KB. On the other hand,
even if it has much more concepts, it is sparse in the knowledge, since it only
uses isA relation (taxonomy). Probase was compared to other taxonomies such as
WordNet, YAGO and Freebase in the sense of recall and precition of isA 
relations. Probase was found to be most comprehensive (biggest recall), while
losing at precision measure against YAGO (92.8\% vs 95\%). 

Probase was later renamed as \emph{Microsoft Concept Graph}, and has accessible
API\footnote{https://concept.research.microsoft.com} which in 2017 consists of
5,401,933 concepts, 12,551,613 instances and 87,603,947 isA relations. 

\subsection {TextRunner}
\label{section:rw_TextRunner}
TextRunner is a successor of KnowItAll system \parencite{Soderland2007} and is
the first to introduce Open Information Extraction (OIE) paradigm, which's main
idea is that it is open-ended and can extract information autonomously without
any human intervention which would fix the system to some specific domain or set
of concepts/relations.
\emph{TextRunner} was ran through over 9 million of web pages, and compared to
\emph{KnowItAll} reduced the error rate for 33\% on comparable set of 
extractions. Throughout the experiments, \emph{TextRunner} collected 11mio
of high probability tuples and 1 mio concrete facts. \emph{TextRunner} consists
of three components.

\emph{Self-Supervised Learner} is a component started first which takes a small
corpus of documents as an input and then outputs a classifier that can detect
candidate extractions and classify them as trustworthy or not.

\emph{Single-Pass Extractor} is a component that makes a single pass through
the full corpus and extracts tuples for all possible relations. These tuples
are then sent to the classifier trained before, which then marks them as
trustworthy or not. Only trustworthy tuples are retained.

\emph{Redundancy-Based Assesor} is the last step which assigns a probability to
each retained tuple, based on the probabilistic model or redundancy
\parencite{Downey2005}.

\subsection{ReVerb}
\label{section:rw_ReVerb}
With  the experiments done with \emph{TextRunner} and \emph{WAE}, it became
obvious that OIE systems have a lot of noise and inconistencies in the results.
For this reason two syntactical and lexical constraints were introduced in
\emph{ReVerb} OIE system\parencite{Fader2011}. This helps with removing
the incoherent extractions such as "recalled began" which was extracted from
sentence "They recalled that Nungesser began his career as precint leader", or
uninformative extractions like "Faust, made a deal" extracted from "Faust
made a deal with the devil".\parencite{Fader2011}.
When started, \emph{ReVerb} first identifies relation phrases that match the
constraints, then if finds approriate pair of appropriate noun phrase rguments
for each identified phrase. The resulting extractions are then given a 
confidence score using logistic regression classifier.

\subsection{R2A2}
\label{section:rw_R2A2}
R2A2 is another improvement in OIE paradigm, since previous systems assumed that
relation arguments are only simple noun phrases. Analysis of \emph{ReVerb}
errors showed that 65\% of errors is on the arguments side (the relation was ok).
To fix this, \emph{R2A2} system goes somehow into the direction of kb based
KA sytems like Curious Cat with argument constraings.\parencite{Etzioni2011}. 
The difference is that 
\emph{R2A2} is not using hard logic and inference, but rather statistical
classifier to detects class constraints (bounds) of the arguments. Compared to
\emph{ReVerb}, \emph{R2A2} has much higher precision and recall.

\subsection{ConceptMiner}
\label{section:rw_ConceptMiner}
\emph{ConceptMiner} is a KA system built by Ian Scott Eslick as part of his
master thesis\parencite{Eslick2006}, with the main hipothesis that the seed
knowledge collected from vounteers can be then used to bootstrap automatic 
knowledge acquisition. \emph{ConceptMiner} specifically focuses on binary
semantic relationships such as cause, effect, intent and time. The system relies
on the prior volunteer knowledge from \emph{ConceptNet} and tests its 
hipothesis with experimental extractions of knowledge aroun three semantic
relations: desire, effect and capability.

As a first step, the systemn uses knowledge around predicates \emph{DesireOf},
\emph{EffectOf} and \emph{CapableOf} from \emph{ConceptNet}, to construct 
web-search queries. The results of these are then used to derive general
patterns for aforementioned relations. For example an existing knowledge
\emph{(DesireOf "dog" "attention")}, when converted to search engine query:
"dog * bark", results in  patterns like:
\begin{itemize}
	\item "My/PRP\$ dog/NN loves/VBZ attention/NN ./."
	\item "Horseback/NN riding /VBG dog /NN attracts/VBZ attention/NN."
\end{itemize}

While not all of the patterns are of the same quality, with the sheer number of
repetitions, it is possible to extract more probable ones. This then results
in general patterns such as \emph{<X>/NN loves/VBZ <Y>/NN}. These can be then
used to issue a lot of search queries with various combinations of wors, to 
extract instances of 'who desires what'. These potential instances then go
into the last step (filtering). As part of this step, \emph{ConceptMiner} 
removes badly formed statements, concepts not included in \emph{ConceptNet} KB,
and concepts with low PMI score (see abbreviations and glossary).

\subsection{DBPedia}
\label{section_rw:dbpedia}
\emph{DBPedia} is crowd-sourced RDF KB, extracted from Wikipedia pages and made
publicly available\parencite{Lehmann2015}. As of 2017 the English DBPedia contains 4.58 million 
knowledge pieces, out of which 4.22 million in a consistent ontology, 
including 1,445,000 persons, 735,000 places, 411,000 creative works
(123,000 music albums, 87,000 films and 19,000 video games), 
241,000 organizations (58,000 companies, 49,000 educational institutions), 
251,000 species and 6,000 diseases\footnote{http://wiki.dbpedia.org/about}.
\emph{DBPedia} is also localized into 125 languages, so all-together it 
consists of 38.3 million knowledge pieces. It is also linked to YAGO categories.

Acquisition mechanism is automatic and consists of the following steps:
\begin{itemize}
\item Wikipedia pages are downloaded from dumps or through API and parsed into
an Abstract Syntax Tree (AST)
\item  AST is forwarded to various extractor modules. For example, extractor
module can find labels, coordinates, etc. Each of the extractor modules can
convert it's part of AST into RDF triples.
\item The collection of RDF statements as returned from the extractors is 
written into an RDF sink, supporting various format such as NTriples, etc.
\end{itemize}

\subsection{YAGO (Yet Another Great Ontology)}
\label{section_rw:YAGO}
\emph{YAGO} is an ontology built automatically from \emph{WordNet} and 
\emph{Wikipedia}\parencite{Suchanek2008}. Latest version \emph{YAGO3} is built
from multiple languages and as of 2015 consist of 4,595,906 entities, 
8,936,324 facts, 15,611,709 taxonimy facts and 1,398,837 labels
\parencite{Mahdisoltani2015}.
The facts were extracted from Wikipedia category system and info boxes, using
a combination of rule-based and heuristic methods, and then enriched with 
hierarchy (taxonomic) relations taken from WordNet. Since building
YAGO is automatized, each next run of the script can use existing knowledge
for type and consistency checking. This kind of type checking helps YAGO to
maintain its precision at 95\%\parencite{Suchanek2008}.

\subsection{KNEXT}
\label{section_rw:KNEXT}
With the premise that there is a lot of general knowledge available in texts,
which lays beneath explicit assertional content, Schubert build \emph{KNEXT} KA
system\parencite{Schubert2002} which extracts \emph{general "possibilistic"
propositions} from text. The main difference towards other KA mining systems is
that before combining meanings from a phrase, the meanings are abstracted
(generalized) and simplified. For example, abstraction of "a long, dak corridor"
yields "a corridor". Or "a small office at the end of a long dark corridor" 
yields "an office". This kind of abstraction, together with weakening of 
relations into a possibilistic form, starts to represent presumptios about the
world. 
The extraction follows five steps:
\begin{itemize}
    \item Pre-process and POS tag the input
    \item Apply a set of ordered patterns to the POS tree recursively
    \item For each successfully matched subtree, abstract the interpretations
          using semantic rule patterns
    \item Collect the phrases expected to hold general "possibilistic" 
        propositions
    \item Formulate the propositiions and output these together with simple
        English representations.
\end{itemize}
From an example input statement "Blanche knew something must be causing 
Stanley's new, strange behavior but she never once connected it with Kitti 
Walker.", the output looks like this:

\lstset{breaklines=true}
\begin{lstlisting}
A female-individual may know a preposition 
(:Q DET FEMALE-INDIVIDUAL) KNOW[V] (:Q DET PROPOS) 
something may cause a behavior
(:F K SOMETHING[N]) CAUSE[V] (:Q THE BEHAVIOR[N])
a male-individual may have a behavior
(:Q DET MALE-INDIVIDUAL) HAVE[V] (:Q DET BEHAVIOR[N])
a behavior can be new
(:Q DET BEHAVIOR[N]) NEW[A])
a behavior can be strange
(:Q DET BEHAVIOR[N]) STRANGE[A])
a female-individual may connect a thing-reffered-to with a female-individual
(:Q DET FEMALE-INDIVIDUAL) CONNECT[V] 
(:Q DET THING-REFERRED-TO) 
(:P WITH[P] (:Q DET FEMALE-INDIVIDUAL)))
\end{lstlisting}

The authors position their system as an addition to systems like \emph{Cyc}, and
conducted their KA experiments on Treebank corpora resulting in around 60\% of
propositions marked as "reasonable general claims"\parencite{Schubert2003}. 

\section{Reasoning Acquisition}
Compared to other types of KA, \emph{Reasoning Acquisition} in its essence,
doesn't need any external data-sources, but it uses existing knowledge and
machine inference to automatically infer additional facts from the existing
knowledge. While deductive reasoning can come with new facts from the premises 
and rules, the reasoning that has a chance to produce higher value (non obvious)
findings is inductive and analogical resasoning. Analogical reasoning can find
new facts of some concept based on properties of similar concepts. On the other
side, infuctive reasoning can find find new probable rules based on current 
observations of the KB.

\subsection {Cyc Predicate Populator + FOIL}
With the initial experiments conducted from Labour Rule Acquisition done as part
of \emph{Factivore} and \emph{Predicate Populator}\parencite{Witbrock2005}, it 
was found that getting inference rules from crowdsourcing and untrained
human labour is ineffective and slow. For this reason, inductive logic 
inference mechanism based on FOIL (First Order Inductive Learner) approach
\parencite{Quinlan1995} was added to the systen. Experiments were conducted on
a set of 10 predicates from the KB, which generated 300 new rules. Of these
rules, 7.5\% were found to be correct and 35\% correct with minor editing to
make them well formed (assertible to Cyc). This way, rule acquisition was speed
up for quite a lot, since previous experiments showed that human experts produce
rules with the rate around three per hour, while with FOIL, they can review
and double check for correctnes around twenty rules per hour.

\subsection {Plausible Inference Patterns (PIP)}

\label{section:rw_PIP}
The main idea of \emph{PIP} system is to learn new plausible inference rules 
(patterns of plausible reasoning) by combining existing knowledge and 
reinforcement learning and thus improve the system's question answering
abilities\parencite{Sharma2010}. It is based on \emph{Cyc} knowledge base,
especially its predicate hierarchy represented by \emph{\#\$genlPred} predicate, 
and \emph{PredicateType} which represents second order collection of predicates 
that can be grouped together by some common features. For example
\emph{(\#\$genlPreds \#\$holds \#\$touches)}, means that each time something is 
holding something else, it is also touching it. The system scans the KB for rules 
containing predicates and tries to generate \emph{PIPs} out of them, meaning
that it tries to replace the predicates in the rules with the appropriate
\emph{PredicateType} which is linked to prior predicate. If there are more
than 5 ways to generate the same PIP (from various predicate instances), then
this new rule is accepted as "valid". Example of such PIP is:
\begin{gather*}
(\#\$familyRelationsSlot(?x,?y) \land \#\$familyRelationSlot(?y,?z)) \\
\implies \\ 
\#\$personalAssociationPredicate(?y,?z)
\end{gather*}

where \emph{\#\$familyRelationsSlot} and \emph{\#\$personalAssociationPredicate}
are instances of \emph{\#\$PredicateType} and thus collections of other 
predicates, and \emph{?x,?y,?z} are variables representing other concepts
\parencite{sharma2010}. This PIP or inference rule, tells the special inference
algorithm that two predicates of type \emph{\#\$familyRelationsSlot} cand
imply \emph{\#\$personalAssociationPredicate}, whent a proper combination
of antecedent bindings on \emph{?x,?y, ?z} is can be proved. Part of the 
\emph{PIP} system is also an inference algorithm (FPEQ), which can make use of 
the above rule and come up with the possible solutions. As the first step, it 
constructs a graph connected to the concepts used in the Q/A query. Then, the
algorithm searches for the assertions in the graph matching a set of existing
PIP inference rules. As the last steps, it returns the maches (i.e. filled-in
PIPs with specific predicates and concepts) which can answer the query.
As an additional step, authors introduuced reinforcement learning, where a small
amount of user feedback (+1 or -1 voting) for final answers, can improve the
PIP selection and also the level of generalizations when generating PIPs.
Overall, the experiments showed that the approach increased Q/A recall for 
quite a lot (120\% improvement), while minimally reduced the precision (94\% of
the baseline)\parencite{Zang2013}.

\subsection{AnalogySpace}
This system is meant to work over big, but unexact knowledge bases 
(such as OMCS), where the need it to be able to make rough conclusions based
on similarities, as opposed on hard and absolute logical facts. 
\emph{AnalogySpace} does just that, by performing analogical closure by
dimensionality reduction of semantic network (KB graph)\parencite{Speer2008}. 
This system tried to represent a new synthesis between a standard symbolic 
reasoning and statistical methods. For this, a similar technique to Latent
Semantic Analysis (LSA) is being used, where strong assertions are used as
opposed to weak semantics of word co-occurences in the document. In the 
LSA matrix, on one axis are concepts from \emph{OMCS}, and on the other
axis a features of these concepts, which yelds a sparse matrix of very
high dimension. Then Singular Value Decomposition (SVD) is used on the matrix
to reduce the dimensionality. This results in \emph{principal components},
which represent the most important aspects of the knowledge. Then semantic
similarity can be determined using linear operations over the resulting vectors.
In a sense, this dimensionality reduction is acting as a generalization process
for the kb. This way it is easier to calculate similarities between resulting
concept vectors and can thus make generalizations based on these similarities,
even if original concept didn't have some of the exact assertions that would
enable inference engine to use it in the inference process and thus come with
good answer. 
Results of experminents have shown that more than 70\% of the resulting 
assertions were marked as true by human validators.

\subsection{Cyc Wiki}
Given that Wikipedia contains a lot of knowledge which is not structured in 
a way to be useful for inference engines directly, it makes sense to either try
to structure it (as was done with \emph{YAGO}), or link it with some existing
ontology such as \emph{CycKB}, which was done as part of this research task
\parencite{Medelyan2008}.
Authors first filtered out of pootential pool of Cyc concepts all non-common 
sense concepts, which ended with 83,897 of them. Then these concepts were
string matched with Wikipedia concept names based on their name directly,
or based on \emph{\#\$nameString} predicate. For example, Cyc concept
\emph{\#\$Virgo-Constellation} would be mathced to Wikipeadia page 
\emph{Virgo (Constellation)}. After this step, the mappings that have 1 to 1
relationship with Wikipedia pages and do not point to Wiki disambiguation pages,
are considered properly aligned. But the mappings that have more than 1 
Wikipedia result then go into the disambiguation phase. With this approach,
authors were able to get the precision of 96.2\% and recall 64.0\% when no 
disambiguation was needed, and precision 93\% and recall 86.3\% with the 
disambiguation part.


\section{Acquistion of Geospatial Context}
This sub section covers the works and approaches that can relate to our 
context mining/acquisition implementation. While the approaches themselves are
more from the data-mining domain, as opposed to knowledge acquisition, in
\emph{Curious Cat}, the results are converted in the knowledge and asserted 
directly as a contextual knowledge about the users. For this reason, this
related work is in its own subsection. In the \tablename~\ref{tab:related}, the 
approaches here are not listed, since they don't count as a knowledge
acquisition, but more a custom data-mining solutions which support our
proposed KA approach.

\subsection{Extracting Places from Traces of Locations}
\label{section:SPD1}
This algorithm (we refer to it as SPD1 - Staypoint Detection), is one of the 
first papers/approaches that started to mine the raw GPS data
into more user friendly notion of location - place.\parencite{Kang2005} The main
idea of the algorithm is to be able to cluster the raw GPS locations into the
particular places (home, work, specific restaurant, etc.) that user visited. 
This is achieved with a combination of radius($D_t$) and time-based ($T_t$) 
clustering tresholds, where a cluster is a set of GPS coordinates that are 
within radius ($dist(loc_1) - dist(loc_n) < D_t$) during a time which is longer 
than a threshold ($time(loc_n)-time(loc_1) > T_t$). With this approach, it is
possible to cluster locations based on how long one stays within a region, 
wihtout knowing the number of clusters in front (as is necessary with more
standard clustering approaches). The core of the algorithm has a benefit
to be really simple and can be easily ran on the phone.

The algorithm pseudocode is shown below (algorithm \ref{alg:spd1}), where $CL$ 
represents currently 
detecting cluster (staypoint), $VP$ is detected cluster (staypoint or Visit 
Point) and $plocs$ is temporal locations to be discarded or added to cluster
later. Regarding the thresholds, $T_t$ is Time threshold, $T_d$ is distance 
treshold.

\begin{algorithm}[htb]
\caption{Staypoint Detection Algorithm 1 (SPD1)}
\label{alg:spd1}

\vspace{5pt}
\KwData{raw GPS coordinates}
\KwResult{cluster representing one staypoint}
\vspace{5pt}
\eIf{distance(CL, loc) < $D_t$}{
    add loc to CL\;
    clear plocs\;
}
{
    \eIf{plocs.length > 1}{
        \If{duration(cl) > $T_t$}{
            add CL to VP\;
        }
        clear CL\;
        add plocs.end to CL\;
        clear plocs\;
        \eIf{distance(CL,loc) < $D_t$}{
            add loc to CL\;
            clear plocs\;
        }{
            add loc to plocs\;
        }
    }{
        add loc to plocs\;
    }
}
\end{algorithm}

While the above alogorithm robustly detects stay-points, it completely ignores
paths between. Additional weak-point is that it doesn't hadndle well when more
than one GPS coordinate wrongly jumps due to GPS accuracy. This was fixed by
the improved algorithm we developed as part of \emph{Curious Cat} system, and
simultanously, the GPS accuracy part as well by the algorithm described 
below (\autoref{section:SPD2}).

\subsection{Discovery of Personal Semantic Places based on Trajectory Data 
Mining}
\label{section:SPD2}
This work\parencite{Lv2016}(SPD2), builds on top of the first \emph{SPD} algoritm 
described in \autoref{section:SPD1}. It improves the stay-point detection and 
incorporates it into the broader system which is able to map particular visits 
into the exact places (points of interest - POIs) and then additionally able to 
predict next locations of tracked users. With this functionality (especially 
mapping to exact points of interest), this approach shares a lot of similarities
with the context mining approach that is employed within the \emph{Curious Cat} 
system. 
The main improvement of this algorithm, is that it takes into the account the
discontinuous characteristics of phone GPS signal (it gets lost inside buildings,
its accuracy drops under the trees). This is done by introducing a second
$T_{dt}$ threshold, which is must be higher than $T_d$, and sets a tolerated
distance between two consequent stay-points to be merged on the fly. 
Additionally, these thresholds are not fixed, but dynamically calculated based
on the distribution of distances between raw GPS points. 

The simplified algorithm (removed unnecessary if/else statements) is shown 
below (algorithm \ref{alg:spd2}), where $CL$
represents currently detecting cluster (staypoint), $PC$ is previous cluster and
$VP$ is detected cluster (staypoint or Visit Point). Regarding the thresholds,
$T_t$ is Time threshold, $T_d$ is distance treshold and $T_{dt}$ is tolerated
distance threshold for later merging.

\begin{algorithm}[htb]
\caption{Staypoint Detection Algorithm 2 (SPD2)}
\label{alg:spd2}

\vspace{5pt}
\KwData{raw GPS coordinates}
\KwResult{cluster representing one staypoint}
\vspace{5pt}
\uIf{distance(CL, loc) < $T_d$}{
    add loc to CL\;
}
\uElseIf{duration(CL)>$T_t$}{
    append CL to VP\;
    CL = 0, PC = 0\;
}
\uElseIf{interval(CL, PC) > $T_t$ {\bf and} distance(CL,PC)< $T_dt$}{
    CL = combine (CL, PC)\;
    append CL to VP\;
    PC = 0\;
}
\Else{
    PC=CL\;
    CL=0\;
}
\end{algorithm}

Similarly as \emph{SPD1}, also \emph{SPD2} has a prblem of ignoring the GPS 
coordinates that are part of paths (or moves) between stay-points.

\subsection{Applying Commonsense Reasoning to Place Identification}
\label{section:MarcoMamei}
Clustering raw sensor data into locations, paths, activities and travel modes
is often not enough. Besides knowing the location, time of arrival and duration
of stay, contextual knowledge benefits from the information about the type of
place and name, or even the exact ID from some database, where additional 
information can be looked up. This is part of the \emph{CuriousCat} system, 
which was inspired by and is based on the related work of Marco 
Mamei\parencite{Mamei2010}.
In this work, the author shows how \emph{Cyc} knowledge base can be used to
improve automatic place identification. The approach is using probabilistic
ranking the list of candidates (retrieved from \emph{Foursquare} in consideration
of the common sense likelihood, which is based on the user profile, features of
the day (exact time, noon, afternoon, morning, midday,...), previous visits
and type of probable place. The approach was validated and given that the
GPS accuracy of N95 phones (used in the experiment) is worse than that of phones
in 2017, it achieved the accuracy of 75\% for business type of places.
